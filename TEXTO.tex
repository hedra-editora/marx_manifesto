\pagestyle{baruch}

\part{Manifesto comunista}

\chapter{Prefácio à edição alemã de 1872}

A Liga dos Comunistas, uma associação operária internacional que, sob as
condições de então, evidentemente só podia ser uma associação secreta,
incumbiu os abaixo"-assinados, no congresso realizado em Londres, em
novembro de 1847, da redação de um detalhado programa teórico e prático
do partido, destinado à publicação. Surgiu assim o \textit{Manifesto}
que se segue, cujo manuscrito se encaminhou para a impressão em Londres
poucas semanas antes da Revolução de Fevereiro. Publicado primeiramente
em alemão, foi reproduzido nesta língua na Alemanha, na Inglaterra e na
América, em pelo menos doze edições diferentes. Em inglês, apareceu
primeiramente no ano de 1850 em Londres, no \textit{Red Republican},
traduzido por Miss Helen MacFarlane, e em 1871 apareceu em pelo menos
três traduções diferentes na América. Em francês, surgiu primeiro em
Paris, pouco antes da insurreição de junho de 1848, e recentemente no
\textit{Le Socialiste} de Nova York. Uma nova tradução está sendo
preparada. Em polonês, apareceu em Londres pouco depois de sua primeira
edição alemã. Em russo, em Genebra, nos anos sessenta. Foi igualmente
traduzido para o dinamarquês pouco depois da sua publicação.

Por mais que as relações tenham se modificado nos últimos 25
anos, os princípios gerais desenvolvidos neste \textit{Manifesto}
conservam, ainda hoje, vistos em conjunto, sua plena justeza. Detalhes
poderiam ser melhorados aqui e ali. A aplicação prática desses
princípios, declara o próprio \textit{Manifesto}, irá depender por toda
parte e a todo tempo das circunstâncias historicamente dadas e, por
isso, não se atribui em absoluto peso especial às medidas
revolucionárias propostas no final do segmento \textsc{ii}. Sob muitos aspectos,
esse passo se formularia hoje de maneira diferente. Em face do imenso
desenvolvimento da grande indústria nos últimos 25 anos e,
ao lado desse desenvolvimento, da crescente organização partidária da
classe operária; em face das experiências práticas, primeiro na
Revolução de Fevereiro e, bem mais ainda, na Comuna de Paris, em que o
proletariado deteve pela primeira vez, ao longo de dois meses, o poder
político, este programa está hoje parcialmente envelhecido. A Comuna,
particularmente, forneceu a prova de que “a classe operária não pode
simplesmente tomar posse da máquina de Estado constituída e colocá"-la
em movimento para os seus próprios objetivos”. (Ver \textit{A guerra
civil na França. Mensagem do Conselho Geral da Associação Internacional
dos Trabalhadores}, onde isso se encontra mais
desenvolvido.) Além disso, é natural que a crítica da literatura
socialista seja lacunar para os dias de hoje, pois chega apenas até o
ano de 1847; igualmente natural que as observações sobre a posição dos
comunistas em relação aos diversos partidos oposicionistas (\mbox{segmento \textsc{iv}}),
se ainda corretas em seus traços fundamentais, já estejam hoje, no entanto, envelhecidas em sua apresentação, uma vez que a situação
política se reconfigurou totalmente e o desenvolvimento histórico
varreu do mapa a maioria dos partidos ali enumerados.

Entretanto, o \textit{Manifesto} é um documento histórico, que não nos
arrogamos mais o direito de modificar. Talvez apareça uma edição
posterior acompanhada de uma introdução que cubra o período de 1847 até
o momento atual; a presente reimpressão pegou"-nos demasiado
desprevenidos para que pudéssemos ter tempo para isso.
\smallskip

\hfill Londres, 24 de junho de 1872

\hfill Karl Marx e Friedrich Engels

\chapter{Prefácio à edição alemã de 1883}

Tenho de assinar sozinho, infelizmente, o prefácio à presente edição.
Marx, o homem a quem toda a classe trabalhadora da Europa e da América
deve mais do que a qualquer outro  ---  Marx descansa no cemitério de
Highgate, e sobre o seu túmulo já cresce a primeira relva. Desde a sua
morte, já não há como falar em refundir ou complementar o
\textit{Manifesto}. Pelo que considero tanto mais necessário registrar
aqui expressamente, mais uma vez, o seguinte.

O pensamento fundamental que atravessa o
\textit{Manifesto} postula que a produção econômica e
a estruturação social de toda época histórica, necessariamente
decorrente daquela, constituem a base da história política e
intelectual dessa época; que, em consonância com isso, toda a história
(desde a dissolução da primitiva propriedade comum da terra e do solo)
tem sido uma história das lutas de classes, lutas entre classes
exploradas e exploradoras, classes dominadas e dominantes, em
diferentes estágios do desenvolvimento social; mas que essas lutas
alcançaram agora um estágio em que a classe explorada e oprimida (o
proletariado) não pode mais se libertar da classe que a explora e
oprime (a burguesia) sem ao mesmo tempo libertar toda a sociedade, para
sempre, da exploração, da opressão e da luta de classes --- este
pensamento fundamental pertence única e exclusivamente a
Marx.\footnote{ “Desse pensamento”, digo eu no prefácio à tradução inglesa, 
“que no meu modo de ver está destinado a fundamentar para a ciência da 
história o mesmo progresso que a teoria de Darwin fundamentou para as 
ciências naturais  ---  desse pensamento, nós dois já nos havíamos paulatinamente 
aproximado alguns anos antes de 1845. Até que ponto eu avançara nessa direção 
por conta própria, mostra"-o o meu \textit{Situação da classe
trabalhadora na Inglaterra}. Mas quando reencontrei Marx em Bruxelas,
na primavera de 1845, ele o tinha formulado de maneira acabada e o
expôs a mim em palavras quase tão claras como estas com que eu o resumi
acima.” [Nota de F.~Engels inserida posteriormente na edição alemã de
1890.]}

Eu já o declarei frequentes vezes; mas justamente agora é necessário que
isso preceda o próprio \mbox{\textit{Manifesto}}.
\smallskip

\hfill Londres, 28 de junho de 1883

\hfill Friedrich Engels

\chapter{Prefácio à edição alemã de 1890}

Desde que as linhas acima foram escritas,\footnote{ Engels refere"-se
ao prefácio para a edição alemã de 1883. [N.~do~T.]}  uma nova
edição alemã do \textit{Manifesto} tornou"-se mais uma vez necessária;
e, desde então, também aconteceram as mais variadas coisas com o
\textit{Manifesto}, as quais cumpre mencionar aqui. 

Uma segunda tradução russa – de autoria de Vera Zassúlitch – apareceu
em Genebra no ano de 1882;\footnote{ Engels se equivoca aqui quanto ao
tradutor desta segunda edição russa do \textit{Manifesto}, confundindo
Vera Ivânovna Zassúlitch com Georgi Plekhanov, ambos membros do grupo
Libertação do trabalho. [N.~do~T.]}  o prefácio para a
mesma foi redigido por Marx e por mim. Infelizmente, o manuscrito
original em alemão desapareceu de minha vista, de modo que tenho de
retraduzir do russo, o que de modo algum será um ganho para o trabalho.
O prefácio diz:

\begin{quote}
Quão limitado era o espaço ocupado então (dezembro de 1847) pelo
movimento proletário, demonstra"-o com máxima clareza o capítulo
conclusivo do \textit{Manifesto}: “Posição dos comunistas em relação
aos diversos partidos oposicionistas”. É que faltam aqui precisamente a
Rússia e os Estados Unidos. Por essa época, a Rússia
representava a última grande reserva da panreação europeia, enquanto
os Estados Unidos absorviam por meio da imigração o excedente de forças
proletárias da Europa. Ambos os países abasteciam a Europa com
matérias"-primas e constituíam, ao mesmo tempo, mercados consumidores
para os seus produtos industrializados. Portanto, ambos os países eram,
então, de uma maneira ou de outra, os alicerces da ordem europeia em
vigor.

Como está tudo diferente hoje! Exatamente, a imigração europeia capacitou
a América do Norte a alcançar uma colossal produção agrícola, cuja
concorrência abala a propriedade fundiária europeia – tanto a grande
quanto a pequena – em seus fundamentos. Além disso, essa imigração
permitiu aos Estados Unidos explorar os seus gigantescos recursos
industriais com uma energia e numa escala que em breve haverão de
romper o monopólio industrial, que vem vigorando até hoje, da Europa
ocidental, em especial o da Inglaterra. Essas duas circunstâncias
retroagem revolucionariamente sobre a própria América. A propriedade
fundiária média e pequena dos \textit{farmers}, base da constituição
política em seu conjunto, vai sucumbindo aos poucos à concorrência das
enormes \textit{farms}; ao mesmo tempo, nos distritos industriais vão
se desenvolvendo pela primeira vez uma massa proletária e uma fabulosa
concentração de capitais. 

E agora a Rússia! Durante a Revolução de 1848 e 1849, não apenas os
príncipes europeus, mas também os burgueses encontraram na intervenção
russa a única salvação diante do proletariado que ia então despertando.
O czar foi proclamado chefe da reação europeia. Hoje ele está como
prisioneiro de guerra da revolução na \textit{Gátchina,}\footnote{
Referência à suntuosa residência imperial na cidade de
\textit{Gátchina}, situada a 40 km ao sul de São Petersburgo. [N.~do~T.]}   e a Rússia representa a vanguarda da ação revolucionária da
Europa.

 O \textit{Manifesto comunista} tinha por tarefa proclamar a dissolução,
inevitavelmente iminente, da moderna propriedade burguesa. Na Rússia,
todavia, em confronto com o engodo capitalista, que floresce rapidamente,
e com a propriedade fundiária burguesa, que só agora se desenvolve,
encontramos mais da metade das terras sob a posse comunitária dos
camponeses. Pergunta"-se então: poderá a
\textit{Ôbschina}\footnote{ \textit{Ôbschina} significa uma
espécie de cooperativa de aldeias na Rússia czarista. Deriva das
palavras \textit{ôbschina} (comunidade, união) e de \textit{ôbschii} (comum, comunitário). [N.~do~T.]} russa
– uma forma, ainda que fortemente solapada, da primitiva posse
comunitária do solo – transitar de modo imediato para a forma mais
elevada da posse comunitária comunista? Ou, ao contrário, ela terá
entes de percorrer o mesmo processo de dissolução que constitui o
desenvolvimento histórico do Ocidente?

 A única resposta que hoje em dia é possível dar a essa questão diz: se
a revolução russa converter"-se em sinal para uma revolução proletária
no Ocidente, de tal modo que ambas se complementem, então a atual
propriedade comum do solo na Rússia poderá contribuir para deflagrar um
desdobramento comunista. 

\hfill Londres, 21 de janeiro de 1882

\hfill Karl Marx e F.~Engels
\end{quote} 

 Uma nova tradução polonesa apareceu por essa mesma época em Genebra:
\textit{Manifest kommunistyczny}.

 Além disso, surgiu uma nova tradução dinamarquesa na
\textit{Socialdemokratisk Bibliotek}, em Copenhague, no ano de 1885.
Infelizmente não é inteiramente completa; algumas passagens
fundamentais, que parecem ter apresentado dificuldades ao tradutor,
foram suprimidas e, ademais, aqui e ali se podem notar vestígios de
desatenção, que ressaltam de maneira tanto mais desagradável quanto se
percebe pelo trabalho que o tradutor poderia ter realizado algo
excelente com um pouco mais de cuidado.

 No ano de 1886 apareceu uma nova tradução francesa no \textit{Le
Socialiste}, em Paris; é a melhor entre todas as traduções publicadas
até hoje.

 Depois desta, publicou"-se, no mesmo ano, uma tradução espanhola,
primeiramente no periódico madrileno \textit{El Socialista} e, depois,
em forma de brochura: \textit{Manifiesto del Partido Comunista}, por
Carlos Marx y F. Engels, Madrid, Administración de “El Socialista”,
Hernán Cortés, 8.

 Menciono ainda, como mera curiosidade, que em 1887 o manuscrito de uma
tradução armênia foi oferecido a um editor de Constantinopla. Mas o bom
homem não teve coragem de imprimir algo que ostentasse o nome de Marx e
ponderou que seria melhor se o próprio tradutor se apresentasse como
autor, coisa que este, todavia, recusou"-se a fazer.

 Depois de se ter impresso\textbf{} repetidamente na Inglaterra esta ou
aquela das traduções americanas mais ou menos incorretas, apareceu, por
fim, uma tradução autêntica no ano de 1888. Ela provém do meu amigo
Samuel Moore e antes da impressão foi revisada mais uma vez por nós
dois. O título é: \textit{Manifesto of the Communist Party}, de Karl
Marx e Frederick Engels. Tradução inglesa autorizada, editada e comentada por Frederick Engels, 1888. Londres, Willian Reeves, 185, Fleet Street, E.~C. Algumas das notas de rodapé dessa edição inglesa foram
incorporadas por mim à presente edição.

 O \textit{Manifesto} teve uma trajetória própria. Saudado
entusiasticamente, quando de seu surgimento, pela então ainda pouco
numerosa vanguarda do socialismo científico (como atestam as traduções
elencadas no primeiro prefácio), ele logo foi constrangido a um segundo
plano pela reação que se iniciou com a derrota dos trabalhadores
parisienses em junho de 1848; e, por fim, proscrito e anatematizado “em
virtude da lei” com a condenação dos comunistas de Colônia em novembro
de 1852. Desaparecendo o movimento operário da cena pública, processo
esse que data da Revolução de Fevereiro, também o \textit{Manifesto}
passou para segundo plano.

 Quando a classe operária europeia se fortaleceu suficientemente para
uma nova investida contra o poderio das classes dominantes, surgiu a
Associação Internacional dos Trabalhadores. Ela tinha por finalidade
fundir todo o operariado militante da Europa e da América num grande
corpo de exército. Por isso, ela não podia tomar como ponto de partida
os princípios estabelecidos no \textit{Manifesto}. Ela precisava ter um
programa que não fechasse as portas às \textit{Trades Unions} inglesas,
aos proudhonianos franceses, belgas, italianos e espanhóis, aos
seguidores alemães de Lassale.\footnote{ Pessoalmente, Lassale sempre se
declarava, perante nós, como “discípulo” de Marx e era óbvio que se
situasse, como tal, no campo do\textit{ Manifesto}. Outra coisa se dava
com aqueles entre os seus adeptos que não iam além de sua reivindicação
de cooperativas de produção com crédito estatal e que dividiam toda a
classe trabalhadora em dois grupos: aqueles que recebem auxílio estatal
e aqueles que dependem apenas de si mesmos. [Nota de F.~Engels.]}  Esse
programa – os considerandos para os estatutos da Internacional – foi
esboçado por Marx com uma mestria reconhecida até mesmo por Bakunin e
pelos anarquistas. Para a vitória final das sentenças estabelecidas no
\textit{Manifesto}, Marx confiava única e exclusivamente no
desenvolvimento intelectual da classe operária, o que necessariamente
deveria decorrer da ação unificada e da discussão. Os acontecimentos e
as vicissitudes na luta contra o capital, as derrotas ainda mais do que
os êxitos, não podiam deixar de revelar aos militantes a insuficiência
das panaceias a que, até então, se apegavam, de abrir suas cabeças para
uma percepção profunda das verdadeiras condições da emancipação dos
trabalhadores. E Marx tinha razão. A classe operária de 1874, quando da
dissolução da Internacional, era completamente diferente daquilo que
fora em 1864, quando de sua fundação. O proudhonismo nos países
românicos, o lassalleanismo específico da Alemanha estavam em extinção,
e mesmo as \textit{Trades Unions} inglesas, então visceralmente
conservadoras, marchavam aos poucos rumo ao ponto em que, no ano de
1887, o presidente de seu congresso em Swansea pôde dizer em nome
delas: “O socialismo continental perdeu, para nós, o seu aspecto
aterrorizador”. Mas já em 1887, o socialismo continental era quase que
tão somente a teoria anunciada no \textit{Manifesto}. E, assim, a
história do \textit{Manifesto} espelha até certo grau a história do
moderno movimento operário a partir de 1848. Nos dias de hoje, é
indubitavelmente o produto mais difundido e mais internacional do
conjunto da literatura socialista, o programa comum de milhões de
trabalhadores de todos os países, da Sibéria até a Califórnia.

  E, contudo, quando o texto apareceu, não poderíamos tê"-lo denominado
manifesto socialista. Em 1847, entendiam"-se por socialistas duas
espécies de pessoas. Por um lado, os adeptos dos diferentes sistemas
utópicos, especialmente os seguidores de Robert Owen, na Inglaterra, e os
de Charles Fourier, na França, sendo que ambos já haviam se reduzido
então à condição de meras seitas, que iam se extinguindo aos poucos.
Por outro lado, os charlatões sociais de todos os matizes que, com as
suas diversas panaceias e com todo tipo de trabalho de remendo, queriam
eliminar as mazelas da sociedade sem causar o menor dano ao capital e
ao lucro. Em ambos os casos, tratava"-se de pessoas que se situavam
fora do movimento operário e, antes, buscavam apoio junto às classes
“cultas”. Em contrapartida, aquela parcela dos trabalhadores que,
convencida da insuficiência de um mero revolucionamento político,
reivindicava uma reconfiguração profunda da sociedade, esta parcela se
denominava então comunista. Era um comunismo apenas toscamente
elaborado, apenas instintivo e, por vezes, um tanto cru; mas já era
suficientemente poderoso para gerar dois sistemas de comunismo utópico:
na França, o “icariano”, de Étienne Cabet e, na Alemanha, o de Wilhelm
Weitling. Em 1847, socialismo significava um movimento burguês enquanto
comunismo, um movimento operário. O socialismo, pelo menos na Europa
continental, era socialmente apresentável, o comunismo era exatamente o
oposto. E uma vez que, já naquela época, tínhamos a firme opinião de
que “a emancipação dos trabalhadores devia ser obra da própria classe
trabalhadora”, não podíamos duvidar nem um só instante sobre qual dos
dois nomes escolher. Desde então, também jamais nos ocorreu
refutá"-lo. 

 “Proletários de todos os países, uni"-vos!” Somente poucas vozes
responderam quando lançamos essas palavras ao mundo, já há 42 anos, às vésperas da primeira revolução parisiense, na qual o
proletariado entrou em cena com reivindicações próprias. Mas, em 28 de
setembro de 1864, proletários da maioria dos países da Europa ocidental
uniam"-se na Associação Internacional dos Trabalhadores, de gloriosa
memória. A Internacional, todavia, viveu apenas nove anos. Mas, que a
eterna aliança dos proletários de todos os países, fundada por ela,
ainda vive, e com mais força do que nunca – para este fato não há
melhor testemunho do que, exatamente, o presente dia. Pois hoje, enquanto
eu escrevo estas linhas, o proletariado europeu e americano passa em
revista suas forças de combate mobilizadas pela primeira vez –
mobilizadas como um exército, sob uma bandeira e para um objetivo
próximo: a implantação legal da jornada de trabalho de oito horas,
proclamada já em 1866 pelo Congresso da Internacional em Genebra e,
reiteradamente, pelo Congresso Operário parisiense em 1889. E o
espetáculo do presente dia haverá de abrir os olhos dos capitalistas e
proprietários fundiários de todos os países para o fato de que hoje os
proletários de todos os países estão efetivamente unidos.

Estivesse Marx ainda ao meu lado, para ver isso com os próprios olhos!
\smallskip

\hfill Londres, 1º de maio de 1890

\hfill Friedrich Engels


\chapter{Manifesto do Partido Comunista}


Um espectro ronda a Europa  ---  o espectro do comunismo. Todas as 
potências da velha Europa aliaram"-se numa sagrada perseguição a esse 
espectro, o papa e o czar, Metternich e Guizot, radicais franceses e policiais alemães.

Onde está o partido de oposição que não tenha sido difamado como
comunista pelos seus adversários governistas, onde está o partido de
oposição que não tenha arremessado de volta, aos opositores mais
progressistas tanto quanto aos seus adversários reacionários, a pecha
estigmatizante do comunismo?

Duas coisas decorrem desse fato.

O comunismo já é reconhecido como uma potência por todas as potências
europeias.

Já é tempo de os comunistas exporem abertamente, perante o mundo
todo, a sua maneira de pensar, os seus objetivos, as suas tendências, e
de contraporem ao conto da carochinha\footnote{ “Conto da carochinha” corresponde no original a \textit{Märchen}, forma
diminutiva do antigo substantivo \textit{Mär} (ou \textit{Märe}), com o
significado, que se constituiu no século \textsc{xv}, de “notícia ou história
inverossímil”. \textit{Märchen} só existe em alto"-alemão, não tendo,
portanto, nenhuma correspondência  nos vários dialetos alemães ou em
qualquer outra língua: \textit{conte de fées}, em francês;
\textit{fairy"-tale}, em inglês; \textit{sprookje}, em holandês. Em
português traduz"-se como conto de “fadas”, “carochinha” ou “conto
maravilhoso”. A palavra ganha notoriedade ao ser empregada pelos irmãos
Grimm como título para a sua coletânea de narrativas populares
publicada entre 1812 e 1815: \textit{Kinder"- und Hausmärchen}
(Contos maravilhosos para crianças e famílias). Menos de quatro     \label{7}
anos antes da redação do \textit{Manifesto}, Heinrich Heine (1797--1856), com quem Marx tivera intensa convivência entre 1843 e 1844,
publica o seu longo poema satírico \textit{Deutschland.~Ein
Wintermärchen} (Alemanha. Um conto maravilhoso de inverno). [N.~do~T.]}
sobre o espectro do comunismo um manifesto do próprio partido.

Com esse objetivo, reuniram"-se em Londres comunistas das mais diversas	
nacionalidades e esboçaram o seguinte manifesto, que está sendo
publicado em idioma inglês, francês, alemão, italiano, flamengo e
dinamarquês.

\section{Burgueses e proletários\protect\footnote{\MakeUppercase{P}or
burguesia entende"-se a classe dos modernos capitalistas, que são os
proprietários dos meios de produção social e exploram o trabalho
assalariado. \MakeUppercase{P}or proletariado, compreende"-se a classe dos modernos
operários assalariados que, uma vez que não possuem meios de produção
próprios, estão na dependência de vender a sua força de trabalho para
poder viver. [\MakeUppercase{N}ota de \MakeUppercase{F.\,E}ngels para a edição inglesa de 1888.]}}


A história de todas as sociedades até o
presente\footnote{ Isto significa, dito de maneira exata, a história legada
pela \textit{escrita.} Em 1847, a pré"-história da sociedade, a organização
social que precedeu toda a história escrita, ainda era praticamente
desconhecida. Desde então, Haxthausen descobriu a propriedade comum do
solo na Rússia, Maurer demonstrou que ela é a base social da qual
derivaram historicamente todas as tribos alemãs, e aos poucos
verificou"-se que comunidades aldeãs com propriedade comum do solo
constituíram a forma primordial da sociedade, da Índia até a Irlanda.
Por fim, a organização interna dessa sociedade comunista primitiva foi
desvendada, em sua forma típica, pela descoberta culminante de Morgan
sobre a verdadeira natureza da \textit{gens} e de sua relação com a
tribo. Com a dissolução desses sistemas comunitários primordiais,
começa a cisão da sociedade em classes especiais e, por fim, em classes
mutuamente opostas. Tentei acompanhar esse processo de dissolução em
minha obra \textit{A origem da família, da propriedade privada e do
Estado}, 2ª edição, Stuttgart, 1886. [Nota de F.~Engels para a edição
inglesa de 1888.]}
é a história das lutas de classes.

Homem livre e escravo, patrício e plebeu, senhor feudal e servo, membro
de corporação e oficial"-artesão, em síntese, opressores e oprimidos
estiveram em constante oposição uns aos outros, travando uma luta
ininterrupta, ora dissimulada, ora aberta, que a cada vez terminava com
uma reconfiguração revolucionária de toda a sociedade ou com a
derrocada comum das classes em luta.

Nas épocas remotas da história, encontramos por quase toda parte
uma estruturação completa da sociedade em diferentes estamentos, uma
gradação multifacetada das posições sociais. Na Roma Antiga temos
patrícios, cavaleiros,\footnote{ Marx e Engels referem"-se aqui à nobreza	
detentora do poder financeiro na antiga Roma, os membros da
“cavalaria” \textit{equites}, que governavam ao lado da nobreza
senatorial. Membros desse estamento da Roma Imperial foram, por
exemplo, os poetas Virgílio e Ovídio. [N.~do~T.]} plebeus, escravos; na 
Idade Média temos senhores feudais, vassalos, membros de corporação, oficiais"-artesãos, 
servos, e ainda, em quase cada uma dessas classes, novas gradações particulares.

A moderna sociedade burguesa, emergente do naufrágio da sociedade
feudal, não aboliu os antagonismos de classes. Ela apenas colocou novas
classes, novas condições de opressão, novas estruturas de luta no lugar
das antigas.

A nossa época, a época da burguesia, caracteriza"-se, contudo, pelo
fato de ter simplificado os antagonismos de classes. A sociedade toda
cinde"-se, mais e mais, em dois grandes campos inimigos, em duas			\label{1}
grandes classes diretamente confrontadas: burguesia e proletariado.

Dos servos da Idade Média advieram os burgueses das   
paliçadas,\footnote{ No original, o termo correspondente a “burgueses das paliçadas” é\label{8}
\textit{Pfahlbürger}, que designa os habitantes, na Baixa Idade Média,
do espaço situado entre as muralhas do castelo e uma circundante
fronteira de paliçada. Em sua condição social, o \textit{Pfahlbürger}
no Medievo alemão corresponde parcialmente ao “vilão” do feudalismo
português. Em sentido figurado, passou a significar, nos séculos
posteriores, uma pessoa demasiado tacanha, de concepções convencionais
e enrijecidas. Com essa mesma conotação metafórica, os termos
\textit{Pfahlbürger} e \textit{Pfahlbürgertum} (“burguesia das
paliçadas”) aparecem, a partir da segunda metade do século \textsc{xix}, em
romances como \textit{O verde Henrique}, do suíço Gottfried Keller, ou
ainda \textit{Buddenbrooks} e \textit{Doutor Fausto}, de Thomas
Mann. [N.~do~T.]}
que habitavam as primeiras cidades; deste estamento medieval 
desenvolveram"-se os primeiros elementos da burguesia.

A descoberta da América, a circum"-navegação da África criaram um novo
campo para a burguesia ascendente. Os mercados das Índias Orientais e
da China, a colonização da América, o intercâmbio com as colônias, a
multiplicação dos meios de troca e das mercadorias em geral deram ao
comércio, à navegação, à indústria um impulso jamais conhecido; e, com
isso, imprimiram um desenvolvimento acelerado ao elemento
revolucionário na sociedade feudal em desagregação.

O funcionamento feudal ou corporativo da indústria, existente
até então, já não bastava para as necessidades que cresciam com os
novos mercados. A manufatura tomou o seu lugar. Os mestres de
corporação foram sufocados\footnote{ Marx e Engels empregam em várias passagens 
desse primeiro segmento o verbo \textit{verdrängen}, que em contexto psicológico 
ou psicanalítico se traduz comumente como ``recalcar''  ---  e enquanto
substantivo (\textit{Verdrängung}), ``recalque''. No sentido em que aparece 
no \textit{Manifesto}, esse verbo alemão pode ser traduzido como
``desalojar'', ``expulsar'', ``tirar do lugar'', ``deslocar'',  ou ainda  --- 
guardando uma relação com a conotação presente em ``recalcar''  --- 
``sufocar'', conforme a opção feita nesta tradução. [N.~do~T.]}
pelo estrato médio industrial; a divisão do trabalho entre as
diversas corporações desapareceu em face da divisão do trabalho no
interior da própria oficina particular.

Mas os mercados continuavam a crescer, continuava a aumentar a
necessidade de produtos. Também a manufatura já não bastava mais. Então
o vapor e a maquinaria revolucionaram a produção industrial. A grande
indústria moderna tomou o lugar da manufatura; o lugar do estrato médio
industrial foi tomado pelos milionários industriais, os chefes de
exércitos industriais inteiros, os burgueses modernos.

A grande indústria criou o mercado mundial, que a descoberta da América
preparara. O mercado mundial deu ao comércio, à navegação, às
comunicações por terra um desenvolvimento incalculável. Este, por sua
vez, reagiu sobre a expansão da indústria, e na mesma medida em que
indústria, comércio, navegação, estradas de ferro se expandiam, nessa
mesma medida a burguesia se desenvolvia, multiplicava os seus capitais,
empurrava a um segundo plano todas as classes provenientes da Idade
Média.

Vemos, portanto, como a própria burguesia moderna é o produto de
um longo processo de desenvolvimento, de uma série de 
revolucionamentos\footnote{ O substantivo ``revolucionamento'' traduz aqui, via de regra, \textit{Umwälzung}, que Marx e Engels empregam várias vezes no
\textit{Manifesto} como espécie de variante de \textit{Revolution}. [N.~do~T.]} nos meios de produção e de transporte.

Cada uma dessas etapas de desenvolvimento da burguesia veio
acompanhada de um progresso político correspondente. Estrato social
oprimido sob o domínio dos senhores feudais, associação armada e com
administração autônoma na comuna;\footnote{ As cidades que iam surgindo na 
França se autodenominavam ``comunas'' mesmo antes de conseguirem arrebatar aos
seus mestres e senhores feudais autoadministração local e direitos políticos 
como ``terceiro Estado''. De forma geral, apresentamos aqui a Inglaterra
como país típico para o desenvolvimento econômico da burguesia; para o seu 
desenvolvimento político, a França. [Nota de F.~Engels para a edição inglesa 
de 1888.]} aqui, cidade"-república independente, ali, terceiro Estado
tributário da monarquia; depois, na era da manufatura, contrapeso à
nobreza na monarquia estamental ou absoluta; base principal das grandes
monarquias de uma forma geral, a burguesia conquistou finalmente para
si, desde a criação da grande indústria e do mercado mundial no moderno
Estado representativo, o domínio político exclusivo. O poder estatal
moderno é apenas uma comissão que administra os negócios comuns do
conjunto da classe burguesa.

A burguesia desempenhou na história um papel extremamente
revolucionário.

Onde quer que a burguesia tenha chegado ao poder, ela destruiu todas as          \label{3}
relações feudais, patriarcais, idílicas. Ela rompeu impiedosamente os
variados laços feudais que atavam o homem ao seu superior natural, não
deixando nenhum outro laço entre os seres humanos senão o interesse nu
e cru, senão o insensível ``pagamento à vista''. Ela afogou os arrepios
sagrados do arroubo religioso, do entusiasmo cavalheiresco, da
plangência do filisteísmo burguês, nas águas gélidas do cálculo
egoísta. Ela dissolveu a dignidade pessoal em valor de troca e, no lugar
das inúmeras liberdades atestadas em documento ou valorosamente
conquistadas, colocou \textit{uma} única inescrupulosa liberdade de
comércio. A burguesia, em uma palavra, colocou no lugar da exploração
ocultada por ilusões religiosas e políticas a exploração aberta,
desavergonhada, direta, seca.

A burguesia despojou de sua auréola sagrada todas as atividades até
então veneráveis, contempladas com piedoso recato. Ela transformou o
médico, o jurista, o clérigo, o poeta, o homem das ciências, em
trabalhadores assalariados, pagos por ela.

A burguesia arrancou às relações familiares o seu comovente véu
sentimental e as reduziu a pura relação monetária.                         

A burguesia revelou como o dispêndio brutal de forças, que a reação
tanto admira na Idade Média, encontrava o seu complemento adequado na
mais indolente ociosidade. Apenas ela deu provas daquilo que a
atividade dos homens é capaz de levar a cabo. Ela realizou obras
miraculosas inteiramente diferentes das pirâmides egípcias, dos
aquedutos romanos e das catedrais góticas, ela executou deslocamentos
inteiramente diferentes das migrações dos povos e das Cruzadas.

A burguesia não pode existir sem revolucionar continuamente os
instrumentos de produção  ---  revolucionar, portanto, as relações de
produção e, assim, o conjunto das relações sociais. Conservação
inalterada do velho modo de produção foi, ao contrário, a condição
primeira de existência de todas as classes industriais anteriores. O
revolucionamento contínuo da produção, o abalo ininterrupto de todas as
situações sociais, a insegurança e a movimentação eternas distinguem a
época burguesa de todas as outras. Todas as relações fixas e
enferrujadas, com o seu séquito de veneráveis representações e
concepções, são dissolvidas; todas as relações novas, posteriormente
formadas, envelhecem antes que possam ossificar"-se. Tudo o que está
estratificado e em vigor
volatiliza"-se,\footnote{ No original alemão, a constatação de que      \label{2}
``tudo o que está estratificado e em vigor volatiliza"-se'' formula"-se		
do seguinte modo: \textit{Alles Ständische und Stehende verdampft}.
Graças ao livro de Marshall Bermann (1982) \textit{All that is Solid
Melts into Air}, difundiu"-se a formulação, referente a esse passo do
\textit{Manifesto}, ``tudo que é sólido desmancha no ar''. Acontece,
porém, que Bermann retirou essas palavras da tradução inglesa realizada
por Samuel Moore, a qual, embora “autorizada” e prefaciada por
Friedrich Engels em janeiro de 1888, simplifica ou traduz com excessiva
liberdade não poucas passagens do \textit{Manifesto}. É o que
também observa Eric Hobsbawn, em sua ``Introdução ao Manifesto
comunista'', em relação à omissão, por S.~Moore, da contração ``nela''  --- 
isto é, na ``revolução''  ---  no fechamento do texto: ``Nela os proletários
nada têm a perder senão as suas cadeias'' (em inglês: \textit{The
proletarians have nothing to lose but their chains}). Hobsbawn observa:
``Embora seja a versão aprovada por Engels, não é uma tradução
rigorosamente correta do texto original''. In \textit{Sobre história},
tradução de Cid Knipel Moreira, São Paulo, Companhia das Letras, 1998,
p.~298. [N.~do~T.]} todo o sagrado é profanado, e os homens são finalmente
obrigados a encarar a sua situação de vida, os seus relacionamentos mútuos,
com olhos sóbrios.

A necessidade de um mercado cada vez mais expansivo para seus produtos
impele a burguesia por todo o globo terrestre. Ela tem de alojar"-se
por toda parte, estabelecer"-se por toda parte, construir vínculos por
toda parte.

Por meio da exploração do mercado mundial, a burguesia       \label{5}
configurou, de maneira cosmopolita, a produção e o consumo de todos os
países. Para grande pesar dos reacionários, ela subtraiu à indústria o
solo nacional em que esta tinha os pés. As antiquíssimas indústrias
nacionais foram aniquiladas e ainda continuam sendo aniquiladas
diariamente. São sufocadas por novas indústrias, cuja introdução se
torna uma questão vital para todas as nações civilizadas, são sufocadas
por indústrias que não mais processam matérias"-primas nativas, mas
sim matérias"-primas próprias das zonas mais afastadas, e cujos
produtos são consumidos não apenas no próprio país, mas simultaneamente
em todas as partes do mundo. No lugar das velhas necessidades,
satisfeitas pelos produtos nacionais, surgem novas necessidades, que
requerem, para a sua satisfação, os produtos dos mais distantes
países e climas. No lugar da velha autossuficiência e do velho
isolamento locais e nacionais, surge um intercâmbio em todas as
direções, uma interdependência múltipla das nações. E o que se dá com a
produção material, dá"-se também com a produção intelectual. Os
produtos intelectuais das nações isoladas tornam"-se patrimônio comum.
A unilateralidade e a estreiteza nacionais tornam"-se cada vez mais
impossíveis, e das muitas literaturas nacionais e locais vai se
formando uma literatura
mundial.\footnote{ Provável referência de Marx e 
Engels ao conceito de literatura mundial (\textit{Weltliteratur}) que
Goethe (1749--1832) desenvolve, a partir de 1827, em ensaios,
resenhas, cartas e conversas. Em várias dessas manifestações, Goethe
também traça paralelos entre a constituição da literatura mundial e a
expansão internacional do comércio. Numa conversa datada de 31 de
janeiro de 1827, Johann Peter Eckermann registra as seguintes palavras
do velho poeta: “Literatura nacional não quer dizer muita coisa agora;
chegou a época da literatura universal (\textit{Weltliteratur}) e cada
um deve atuar no sentido de acelerar essa época”.  [N.~do~T.]}

Mediante o rápido aperfeiçoamento de todos os instrumentos de produção,
por meio das comunicações infinitamente facilitadas, a burguesia
arrasta todas as nações, mesmo as mais bárbaras, para dentro da
civilização. Os módicos preços de suas mercadorias são a artilharia
pesada com que ela põe abaixo todas as muralhas da China, com que ela
constrange à capitulação mesmo a mais obstinada xenofobia dos bárbaros.
Ela obriga todas as nações que não queiram desmoronar a apropriar"-se
do modo de produção da burguesia; ela as obriga a introduzir em seu
próprio meio a assim chamada civilização, isto é, a tornarem"-se
burguesas. Em uma palavra, ela cria para si um mundo à sua própria
imagem e semelhança.\footnote{Num hino do jovem Goethe muito admirado por Marx (\textit{Prometeu}, escrito em 1774), o titã que rouba o fogo dos deuses para doá-lo aos mortais afirma na última estrofe que forma as pessoas à sua própria imagem e semelhança.}

A burguesia submeteu o campo ao domínio da cidade. Ela criou
cidades enormes, aumentou o número da população urbana, em face da
rural, em alta escala e, assim, arrancou do
idiotismo\footnote{ \textit{Idiotismus}, no original. Na mencionada “Introdução
ao Manifesto comunista”, E. Hobsbawn observa quanto a essa formulação que,
embora os autores do \textit{Manifesto} partilhassem do costumeiro
desprezo do citadino pelo mundo rural (mas também da ignorância do
citadino em relação a esse mundo), o termo \textit{Idiotismus} possui
antes o sentido de ``horizontes estreitos'' do que de ``estupidez'': ``fazia
eco ao sentido original do termo grego \textit{idiot\=es}, do qual derivou o
significado corrente de `idiota' ou `idiotice', a saber, uma `pessoa
preocupada apenas com seus próprios assuntos particulares e não com os
da comunidade mais ampla'. No curso das décadas posteriores a 1840, e
em movimentos cujos membros, ao contrário de Marx, não possuíam
educação clássica, o sentido original se evaporou ou foi mal
interpretado'' (\textit{Op.~cit.}, p.~298). [N.~do~T.]}
da vida rural uma parcela significativa da população. Da mesma
forma como torna o campo dependente da cidade, ela torna os países
bárbaros e semibárbaros dependentes dos civilizados, os povos agrários
dependentes dos povos burgueses, o Oriente dependente do Ocidente.

A burguesia vem abolindo cada vez mais a fragmentação dos meios de
produção, de propriedade e também a fragmentação da população. Ela
aglomerou a população, centralizou os meios de produção e concentrou a
propriedade em poucas mãos. Consequência necessária disso tudo foi a
centralização política. Províncias independentes, quase que
tão somente aliadas, com interesses, leis, governos e sistemas
aduaneiros diversificados, foram aglutinadas em \textit{uma} nação,
\textit{um} governo, \textit{um} interesse nacional de classe,
\textit{uma} fronteira aduaneira.

Em seu domínio de classe que mal chega a um século, a burguesia criou
forças produtivas em massa, mais colossais do que todas as gerações
passadas em seu conjunto. Subjugação das forças da natureza,
maquinaria, aplicação da química na indústria e na agricultura,
navegação a vapor, estradas de ferro, telégrafos elétricos,
arroteamento de continentes inteiros, canalização dos rios para a
navegação, populações inteiras como que brotando do chão  ---  que século
passado poderia supor que tamanhas forças produtivas estivessem
adormecidas no seio do trabalho social!

Nós vimos, portanto: os meios de produção e de circulação, sobre cujas
bases a burguesia se formou, foram gerados no âmbito da sociedade
feudal. Num certo estágio do desenvolvimento desses meios de produção e
de circulação, as relações nas quais a sociedade feudal produzia e
trocava, a organização feudal da agricultura e da manufatura, em uma
palavra, as relações feudais de propriedade, não correspondiam mais às
forças produtivas já desenvolvidas. Elas tolhiam a produção, em vez de
fomentá"-la. Transformavam"-se assim em outros tantos grilhões.
Precisavam ser explodidas e foram explodidas.

Em seu lugar entrou a livre concorrência, com a constituição social e
política que lhe era adequada, com o domínio econômico e político da
classe burguesa.

Sob os nossos olhos processa"-se um movimento semelhante. As relações
burguesas de produção e de circulação, as relações burguesas de
propriedade, a moderna sociedade burguesa, que fez aparecer meios de
produção e de circulação tão poderosos, assemelha"-se ao feiticeiro
que já não consegue mais dominar os poderes subterrâneos que invocou.\footnote{Alusão à balada de Goethe \textit{O aprendiz de feiticeiro} (1797), cujo motivo da perda de controle sobre o feitiço aparece também em Luciano de Samósata (século \textsc{ii} d.C.).}
Há decênios a história da indústria e do comércio vem sendo apenas a
história da revolta das modernas forças produtivas contra as modernas
relações de produção, contra as relações de propriedade que constituem
as condições vitais da burguesia e de sua dominação. Basta mencionar as
crises comerciais que, em sua recorrência periódica, questionam de
maneira cada vez mais ameaçadora a existência de toda a sociedade
burguesa. Nas crises comerciais extermina"-se regularmente não apenas
uma grande parte dos produtos fabricados, mas também das forças
produtivas já criadas. Deflagra"-se nas crises uma epidemia social que
a todas as épocas anteriores apareceria como contrassenso  ---  a
epidemia da superprodução. A sociedade encontra"-se remetida
subitamente a um estado de momentânea barbárie; uma epidemia de fome,
uma guerra geral de extermínio parecem ter"-lhe cortado todo
suprimento de alimentos; a indústria, o comércio parecem aniquilados  --- 
e por quê? Porque a sociedade possui demasiada civilização, demasiados
suprimentos de alimentos, demasiada indústria, demasiado comércio. As
forças produtivas que estão à sua disposição já não servem mais ao
fomento das relações de propriedade burguesas; ao contrário, elas se
tornaram por demais poderosas para essas relações, são tolhidas por
elas; e tão logo superam esse obstáculo, levam toda a sociedade
burguesa à desordem, põem em perigo a existência da propriedade
burguesa. As relações burguesas tornaram"-se demasiado estreitas para
abarcar a riqueza gerada por elas. --- Por quais meios a burguesia supera    \label{9}
as crises? Por um lado, pelo extermínio forçado de grande parte das
forças produtivas; por outro lado, pela conquista de novos mercados e
pela exploração mais metódica dos antigos mercados. Como isso acontece
então? Pelo fato de que a burguesia prepara crises cada vez mais amplas
e poderosas, e reduz os meios de preveni"-las.

As armas com as quais a burguesia derruiu o feudalismo voltam"-se agora     \label{4}
contra a própria burguesia.

Mas a burguesia não forjou apenas as armas que lhe trazem a morte; ela
produziu também os homens que portarão essas armas  ---  os operários
modernos, os \textit{proletários}.

Na mesma medida em que a burguesia, isto é, o capital, desenvolve"-se,
desenvolve"-se também o proletariado, a classe dos modernos operários,
os quais só subsistem enquanto encontram trabalho, e só encontram
trabalho enquanto o seu trabalho aumenta o capital. Esses operários,
que têm de se vender um a um, são uma mercadoria como qualquer outro
artigo de comércio e, por isso, igualmente expostos a todas as
vicissitudes da concorrência, a todas as oscilações do mercado.

O trabalho dos proletários perdeu, pela expansão da maquinaria e
pela divisão do trabalho, todo caráter autônomo e, com isso, todo
atrativo para o operário. Este torna"-se um mero acessório da máquina,
do qual é exigido apenas o mais simples movimento de mãos, o mais
monótono, o mais fácil de aprender. Os custos que o operário causa
restringem"-se, por isso, quase que tão somente aos alimentos de que
ele carece para o sustento próprio e para a reprodução de sua
raça.\footnote{ \textit{Rasse}, no original. [N.~do~T.]}
Mas o preço de uma mercadoria, portanto também do trabalho, é
igual aos seus custos de produção. Na mesma medida em que cresce o
caráter repugnante do trabalho, diminui por isso mesmo o salário. Mais
ainda, na mesma medida em que a maquinaria e a divisão do trabalho
aumentam, aumenta a massa do trabalho, seja pela multiplicação das
horas de trabalho, seja pela multiplicação do trabalho exigido num
espaço de tempo determinado, seja pelo funcionamento acelerado da
máquina etc.


A indústria moderna transformou a pequena oficina do mestre patriarcal
na grande fábrica do capitalista industrial. Massas de operários,
aglomeradas na fábrica, são organizadas de forma soldadesca. Como
soldados rasos da indústria, são colocados sob a supervisão de uma
hierarquia completa de suboficiais e oficiais. Eles não apenas são
servos da classe burguesa, do Estado burguês; diariamente e a cada hora
eles são escravizados pela máquina, pelo supervisor e, sobretudo, por
cada um dos fabricantes burgueses. Esse despotismo é tanto mais
mesquinho, odioso, encarniçado, quanto mais abertamente ele proclama o
lucro como o seu objetivo.


Quanto menos o trabalho manual requer habilidade e dispêndio de forças,
isto é, quanto mais a indústria moderna se desenvolve, tanto mais o
trabalho dos homens é sufocado pelo das mulheres. Diferenças de sexo e
de idade não têm mais qualquer validade social para a classe operária.
Só restam ainda instrumentos de trabalho que, de acordo com idade e
sexo, perfazem custos variados.

Se a exploração do operário pelo fabricante está terminada no momento em
que aquele recebe o seu salário em dinheiro vivo, abatem"-se sobre ele
então as outras parcelas da burguesia, o proprietário do imóvel, o dono
da mercearia, o penhorista etc.

Os pequenos estratos médios até hoje existentes, os pequenos
industriais, os comerciantes e os que vivem de pequenas rendas, os
artesãos e os camponeses, todas essas classes decaem no proletariado,
em parte porque o seu pequeno capital não basta para o grande
empreendimento industrial e sucumbe à concorrência com os capitalistas
maiores, em parte porque a sua habilidade é desvalorizada pelos novos
modos de produção. Assim, o proletariado é recrutado de todas as classes
da população.

O proletariado atravessa diversas etapas de desenvolvimento. A sua luta
contra a burguesia começa com a sua existência.

No início lutam os operários isolados, depois os operários de uma   \label{6}
fábrica, depois os operários de um ramo industrial, numa mesma região,
contra um burguês particular, que os explora diretamente. Eles dirigem
os seus ataques não apenas contra as relações de produção burguesas;
eles os dirigem contra os próprios instrumentos de produção; eles
aniquilam as mercadorias estrangeiras concorrentes, destroçam as
máquinas, ateiam fogo nas fábricas, buscam reconquistar a soterrada
posição do trabalhador medieval.

Nessa etapa, os operários formam uma massa dispersa por todo o país e
fragmentada pela concorrência. A agregação em massa dos operários ainda
não é a consequência de sua própria associação, mas sim a consequência
da associação da burguesia que, para alcançar seus próprios objetivos
políticos, tem de mobilizar todo o proletariado, o que por enquanto el
ainda consegue. Nessa etapa, portanto, os proletários combatem não os
seus inimigos, mas sim os inimigos de seus inimigos, os resquícios da
monarquia absoluta, os proprietários de grandes territórios, os
burgueses não industriais, os pequeno"-burgueses. Toda a movimentação
histórica está concentrada assim nas mãos da burguesia; toda vitória
assim conquistada é uma vitória da burguesia.

Mas com o desenvolvimento da indústria, não apenas se multiplica o
proletariado; este é agregado em massas cada vez maiores, sua força
cresce e torna"-se mais perceptível para ele mesmo. Os interesses, as
situações de vida no interior do proletariado nivelam"-se cada vez
mais, à medida que a maquinaria dissipa cada vez mais as diferenças do
trabalho e, por quase toda parte, comprime o salário para um nível
igualmente baixo. A crescente concorrência entre os burgueses e as
crises de comércio daí resultantes fazem o salário do operário oscilar
cada vez mais; o aperfeiçoamento incessante da maquinaria,
desenvolvendo"-se com crescente rapidez, torna cada vez mais insegura
toda a sua condição de vida; cada vez mais, as colisões entre o
operário particular e o burguês particular assumem o caráter de
colisões entre duas classes. Os operários começam a constituir
coalizões contra os burgueses; eles congregam"-se para a garantia de
seus salários. Chegam mesmo a fundar associações permanentes com a
finalidade de criar provisões de mantimentos para eventuais revoltas.
Aqui e acolá, a luta eclode em sublevação.

De tempos em tempos triunfam os operários, mas apenas provisoriamente. O
resultado efetivo de suas lutas não é o êxito imediato, mas sim uma
união operária em crescente expansão. Ela é fomentada pelos meios de
comunicação que, gerados pela grande indústria, avolumam"-se e
colocam os operários das diversas localidades em contato mútuo. O mero
contato, porém, basta para centralizar as muitas lutas locais, com
caráter semelhante por toda parte, em uma luta nacional, uma luta de
classes. Mas toda luta de classes é uma luta política. E a união, para
a qual os burgueses da Idade Média, com seus caminhos vicinais,
necessitaram de séculos, os proletários modernos, com as estradas de
ferro, executam"-na em poucos anos.

Essa organização dos proletários em classe e, com isso, em partido
político, é a todo momento rompida pela concorrência entre os próprios
operários. Mas ela ressurge sempre de novo, mais forte, mais sólida,
mais poderosa. Ela impõe o reconhecimento de interesses particulares
dos operários em forma de lei, à medida que se aproveita das cisões
internas da burguesia. É o caso da lei da jornada de dez horas, na
Inglaterra.

As colisões no interior da velha sociedade promovem em geral, de
múltiplos modos, o processo de desenvolvimento do proletariado. A
burguesia encontra"-se em luta contínua: no início, contra a
aristocracia; mais tarde, contra as frações da própria burguesia cujos
interesses entraram em contradição com o progresso da indústria; e,
sempre, contra a burguesia de todos os países estrangeiros. Em todas
essas lutas, ela se vê obrigada a apelar ao proletariado, a reivindicar
a sua ajuda e, assim, a engolfá"-lo no movimento político. Ela mesma,
portanto, leva ao proletariado os seus próprios elementos de
formação, \footnote{ Na edição alemã publicada em 1888 lê"-se: ``os elementos	
de sua própria formação política e geral''. [N.~do~T.]} isto é, armas contra 
si mesma.

Além disso, como vimos, contingentes inteiros da classe
dominante são arrastados para o proletariado em virtude do progresso da
indústria, ou pelo menos ameaçados em suas condições de vida.
Também esses contingentes levam ao proletariado grande quantidade de
elementos de
formação.\footnote{ Na edição de 1888: ``elementos de esclarecimento e de 
progresso''. [N.~do~T.]}

Em tempos, por fim, em que a luta de classes se aproxima da decisão, o
processo de dissolução no interior da classe dominante, no interior de
toda a velha sociedade, assume um caráter tão violento, tão estridente,
que uma pequena fração da classe dominante se desliga dela e se associa
à classe revolucionária, à classe que traz o futuro em suas mãos. Por
isso, assim como outrora uma parcela da nobreza passou para o lado da
burguesia, uma parcela da burguesia passa agora para o proletariado, e
notadamente uma parcela dos ideólogos burgueses que se alçaram à
compreensão teórica do movimento histórico em sua totalidade.

De todas as classes que se defrontam hoje com a burguesia, somente o
proletariado é uma classe realmente revolucionária. As classes
restantes vão se degenerando e afundam sob a grande indústria; o
proletariado é o seu produto mais genuíno.

Os estratos médios, o pequeno industrial, o pequeno comerciante, o
artesão, o camponês, todos eles combatem a burguesia para salvar do naufrágio sua
existência, como estratos médios. Eles, portanto, não
são revolucionários, mas sim conservadores. Mais ainda, são
reacionários, procuram girar para trás a roda da história. Se eles são
revolucionários, então só o são com vistas à sua passagem iminente para
o proletariado, e assim defendem não os seus interesses atuais, mas os
futuros, abandonando assim a sua posição própria para colocarem"-se na
posição do proletariado.

O lumpemproletariado, esse apodrecimento passivo das camadas mais baixas
da velha sociedade, é parcialmente arrastado para o movimento por obra
de uma revolução proletária; mas em consonância com toda a sua situação
de vida, ele estará mais pronto a se vender para maquinações
reacionárias.

As condições de vida da velha sociedade já estão aniquiladas nas
condições de vida do proletário. O proletariado é despossuído; sua
relação com mulher e filhos não tem nada mais em comum com a relação
familiar burguesa; o moderno trabalho industrial, a moderna subjugação
operada pelo capital, na Inglaterra a mesma que na França, na América a
mesma que na Alemanha, despojou o proletário de todo caráter nacional.
As leis, a moral, a religião são para ele outros tantos preconceitos
burgueses atrás dos quais se escondem outros tantos interesses
burgueses.

Todas as classes anteriores que conquistaram o poder para si, procuraram
assegurar sua condição de vida já adquirida à medida que submetiam toda
a sociedade às condições de sua aquisição. Os proletários só podem
conquistar as forças produtivas sociais à medida que abolem o seu
próprio modo de apropriação e, assim, todo o modo de apropriação até
hoje existente. Os proletários não possuem nada de próprio para
assegurar, eles têm de destruir todas as seguranças privadas e todas as
garantias privadas até hoje existentes.

Todos os movimentos até o presente foram movimentos de minorias ou em
proveito de minorias. O movimento proletário é o movimento autônomo da
maioria esmagadora em proveito da maioria esmagadora. O proletariado, a
camada mais baixa da sociedade atual, não pode erguer"-se,
aprumar"-se, sem que vá para os ares toda a superestrutura dos
estamentos que formam a sociedade oficial.

Ainda que não pelo conteúdo, pela forma a luta do proletariado contra a
burguesia é primeiramente uma luta nacional. O proletariado de todo e
qualquer país tem primeiro, naturalmente, de dar conta de sua própria
burguesia.

Na medida em que traçamos as fases mais gerais do desenvolvimento do
proletariado, acompanhamos a guerra civil, a qual se desenrola de forma
mais ou menos oculta no interior da sociedade em vigor, até o ponto em
que eclode em uma revolução aberta e, pela derrubada violenta da
burguesia, o proletariado estabelece a sua dominação.

Toda a sociedade até hoje existente assentou"-se, como vimos,
no antagonismo de classes opressoras e oprimidas. Mas para que se possa
oprimir uma classe é necessário assegurar"-lhe condições em cujo
âmbito ela consiga ao menos manter sua existência servil. O servo
alçou"-se a membro da comuna durante a servidão, assim como o
pequeno"-burguês alçou"-se à condição de burguês sob o jugo do
absolutismo feudal. O operário moderno, ao contrário, em vez de
elevar"-se com o progresso da indústria, vai caindo cada vez mais
fundo, abaixo das condições de sua própria classe. O operário
torna"-se um pauperizado, e o
pauperismo\footnote{ Os termos ``pauperizado'' e ``pauperismo'' correspondem, 
no original, a \textit{Pauper} e \textit{Pauperismus}. A esse respeito observa 
Hobsbawn numa nota à sua ``Introdução'': ``Pauperismo não deve ser lido
como sinônimo de ‘pobreza’. As palavras alemãs, emprestadas do uso
inglês, são ‘Pauper’, ‘uma pessoa destituída [\ldots] mantida pela
caridade ou por algum fundo público’ (\textit{Chambers Twentieth
Century Dictionary}), e ‘Pauperismus’ (pauperismo: ‘condição de ser
indigente’, \textit{ibid}.).'' [N.~do~T.]}
desenvolve"-se ainda mais depressa do que a população e a
riqueza. Com isso, torna"-se evidente que a burguesia é incapaz de
permanecer por mais tempo como a classe dominante da sociedade e de
impor à sociedade, como lei reguladora, as condições de vida de sua
classe. Ela é incapaz de dominar porque é incapaz de assegurar aos seus
escravos uma existência mesmo no âmbito da escravidão, porque ela é
obrigada a deixá"-los descer a uma situação em que ela tem de
alimentá"-los, em vez de ser alimentada por eles. A sociedade não pode
mais viver sob a burguesia, isto é, a vida desta não é mais compatível
com a sociedade.

A condição essencial para a existência e para a dominação da classe
burguesa é a acumulação da riqueza em mãos privadas, a formação e a
multiplicação do capital; a condição do capital é o trabalho
assalariado. O trabalho assalariado assenta"-se exclusivamente sobre a
concorrência dos operários entre si. O progresso da indústria, de que a
burguesia é a representante indolente e apática, substitui o isolamento
dos operários, que se dá através da concorrência, pela sua união
revolucionária mediante a associação. Com o desenvolvimento da grande
indústria, subtrai"-se, portanto, à burguesia a própria base sobre a
qual ela produz e apropria"-se dos produtos. Ela produz em primeiro
lugar o seu próprio coveiro. A sua derrocada e a vitória do
proletariado são igualmente inevitáveis.

\section{Proletários e comunistas}

De que forma os comunistas se relacionam com os proletários em geral?

Os comunistas não constituem, em face dos outros partidos operários,
nenhum partido particular.

Eles não possuem interesses separados dos interesses do conjunto do
proletariado.

Eles não sustentam princípios particulares, de acordo com os quais
queiram moldar o movimento proletário.

Por um lado, os comunistas só se diferenciam dos demais partidos
proletários pelo fato de enfatizarem e fazerem prevalecer, nas várias
lutas nacionais dos proletários, os interesses comuns de todo o
proletariado, independentemente de nacionalidade; e, por outro lado,
pelo fato de sempre representarem, nas diversas etapas de
desenvolvimento por que passa a luta entre proletariado e burguesia, os
interesses do movimento em seu conjunto.

Os comunistas são assim, na prática, a fração mais decidida dos partidos
operários de todos os países, a qual sempre impulsiona o movimento para
diante; na teoria, eles têm de vantagem sobre a massa restante do
proletariado a percepção consciente das condições, da marcha e dos
resultados gerais do movimento proletário.

O objetivo mais próximo dos comunistas é o mesmo de todos os demais
partidos proletários: formação do proletariado em classe, derrubada da
dominação burguesa, conquista do poder político pelo proletariado.

As proposições teóricas dos comunistas não se baseiam de forma alguma em
ideias, em princípios inventados ou descobertos por esse ou aquele
reformador do mundo.

Elas são apenas expressões gerais de relações efetivas de uma luta de
classes existente, expressões de um movimento histórico que se
desenrola sob os nossos olhos. A abolição das relações de propriedade
até hoje em vigor não é nada que assinale o comunismo de maneira
peculiar.

Todas as relações de propriedade estiveram submetidas a uma constante
mudança histórica, a uma constante transformação histórica.

A Revolução Francesa, por exemplo, aboliu a propriedade feudal em
benefício da propriedade burguesa.

O que distingue o comunismo não é a abolição da propriedade em geral,
mas sim a abolição da propriedade burguesa.

Mas a moderna propriedade privada burguesa é a expressão última e mais
acabada do modo de produção e de apropriação de produtos que repousa em
antagonismos de classes, na exploração de umas classes pelas outras.

Nesse sentido, os comunistas podem resumir a sua teoria na única
expressão: supressão da propriedade privada.

Censuraram a nós, comunistas, querer abolir a propriedade adquirida de
forma pessoal, fruto do próprio trabalho; abolir a propriedade que
constitui a base de toda liberdade, atividade e autonomia pessoais.

Propriedade adquirida, fruto do próprio trabalho e do mérito! Vocês
estão falando da propriedade do pequeno"-burguês, do pequeno camponês,
a qual precedeu a propriedade burguesa? Nós não precisamos aboli"-la,
o desenvolvimento da indústria aboliu"-a e vai abolindo"-a
diariamente.

Ou vocês estão falando da moderna propriedade privada burguesa?

Mas o trabalho assalariado, o trabalho do proletário, cria"-lhe
propriedade? De forma alguma. Ele cria o capital, isto é, a propriedade
que explora o trabalho assalariado, propriedade que só pode
multiplicar"-se sob a condição de produzir novo trabalho assalariado
para explorá"-lo renovadamente. Em sua forma atual, a propriedade
move"-se no interior do antagonismo entre capital e trabalho
assalariado. Contemplemos os dois lados desse antagonismo.

Ser capitalista significa assumir não apenas uma posição meramente
pessoal na produção, mas também uma posição social. O capital é um
produto coletivo e só pode ser posto em movimento mediante a atividade
comum de muitos membros, e até mesmo, em última instância, mediante a
atividade comum de todos os membros da sociedade.

O capital, portanto, não é uma potência pessoal, ele é uma potência
social.

Desse modo, ao transformar"-se o capital em propriedade coletiva,
pertencente a todos os membros da sociedade, então não é a propriedade
pessoal que se transforma em coletiva. Transforma"-se apenas o caráter
social da propriedade. Esta perde o seu caráter de classe.

Passemos ao trabalho assalariado.

O preço médio do trabalho assalariado é o mínimo do salário de trabalho,
isto é, a soma dos meios de subsistência que são necessários para
manter a vida do operário enquanto operário. Aquilo, portanto, de que o
operário assalariado se apropria mediante a sua atividade, é suficiente
tão somente para reproduzir a sua vida pura e simples. Nós não
queremos de forma alguma abolir essa apropriação pessoal dos produtos
do trabalho para a reprodução da vida imediata, uma apropriação que não
deixa nenhum lucro líquido que poderia conferir poder sobre trabalho
alheio. Queremos apenas suprimir o caráter miserável dessa apropriação,
na qual o operário vive apenas para multiplicar o capital, e vive
tão somente enquanto o requer o interesse da classe dominante.

Na sociedade burguesa, o trabalho vivo é apenas um meio de multiplicar o
trabalho acumulado. Na sociedade comunista, o trabalho acumulado é
apenas um meio para ampliar, enriquecer, fomentar o processo de vida do
operário.

Na sociedade burguesa, o passado impera, portanto, sobre o presente; na
comunista, é o presente que impera sobre o passado. Na sociedade
burguesa, o capital é autônomo e pessoal, enquanto que o indivíduo ativo
é impessoal e privado de autonomia.

E à supressão dessa relação a burguesia chama supressão da personalidade
e da liberdade! E com razão. Trata"-se, todavia, da supressão da
personalidade, da autonomia e da liberdade dos burgueses.

Por liberdade entende"-se, no âmbito das atuais relações de produção
burguesas, o livre comércio, a livre compra e venda.

Mas se cai a barganha, então cai também a barganha livre. De uma maneira
geral, todo o palavrório referente à livre barganha, como todas as
demais bravatas de nossa burguesia sobre a liberdade, só fazem sentido
em face da barganha tolhida, do burguês subjugado da Idade Média, mas
não em face da supressão comunista da barganha, das relações burguesas
de produção e da própria burguesia.

Vocês se horrorizam com o fato de querermos suprimir a propriedade
privada. Mas na sociedade vigente, na sociedade de vocês, a propriedade
privada está abolida para nove décimos de seus membros; ela existe
exatamente por não existir para nove décimos. Vocês, portanto,
censuram"-nos querer suprimir uma propriedade que pressupõe, como
condição necessária, a privação de propriedade para a maioria
esmagadora da sociedade.

Vocês nos censuram, em uma palavra, querer suprimir a propriedade de
vocês. Entretanto, é isso mesmo que queremos.

A partir do momento em que o trabalho não possa mais ser transformado em
capital, em dinheiro, em renda fundiária, em suma, em uma potência social
monopolizável, isto é, a partir do momento em que a propriedade pessoal
não possa mais reverter em propriedade burguesa, a partir desse
momento, declaram vocês, a pessoa estaria suprimida.

Vocês confessam, portanto, não conceber sob a condição de pessoa nada
além do burguês, do proprietário burguês. E essa pessoa, todavia,
precisa ser suprimida.

O comunismo não tira de ninguém o poder de apropriar"-se de produtos
sociais, ele apenas tira o poder de subjugar o trabalho alheio mediante
essa apropriação.

Objetou"-se que, com a supressão da propriedade privada, cessaria toda
atividade e irromperia uma indolência geral.

De acordo com isso, a sociedade burguesa deveria ter perecido há muito
tempo na indolência; pois os que nela trabalham, não lucram, e os que
nela lucram, não trabalham. Todo esse escrúpulo converge para a
tautologia de que não mais existirá trabalho assalariado tão logo não
exista mais capital.

Todas as investidas, que são dirigidas ao modo comunista de apropriação
e de produção dos produtos materiais foram igualmente estendidas à
apropriação e à produção dos produtos intelectuais. Da mesma maneira
como, para o burguês, o cessamento da propriedade de classe significa o
cessamento da própria produção, o cessamento da formação de
classe é, para ele, idêntico ao cessamento da formação cultural de uma
forma geral.

A formação cultural, cuja perda ela lamenta, é, para a imensa maioria, a
formação direcionada para a máquina.

Mas não venham discutir conosco enquanto avaliarem a abolição da
propriedade burguesa com a medida das suas representações burguesas de
liberdade, formação, direito etc. As próprias ideias de vocês são
produtos das relações burguesas de produção e propriedade, como o
sistema jurídico de vocês é apenas a vontade de sua classe elevada à
condição de lei, uma vontade cujo conteúdo está dado nas condições
materiais de vida da classe de vocês.

A representação interessada, que os leva a transformar as suas relações
de produção e de propriedade  ---  de relações históricas, transitórias no
desenrolar da produção, em leis eternas da natureza e da razão  ---, vocês
a partilham com todas as classes dominantes desaparecidas. O que vocês
compreendem em relação à propriedade antiga, o que compreendem em
relação à propriedade feudal, vocês não conseguem mais compreender em
relação à propriedade burguesa.

Supressão da família! Mesmo os mais radicais exaltam"-se em face desse
infame desígnio dos comunistas.

Sobre o que repousa a família atual, a família burguesa? Sobre o
capital, sobre o lucro privado. Somente para a burguesia a família
existe de forma plenamente desenvolvida; mas ela encontra o seu
complemento na carência de família imposta aos proletários e na
prostituição pública.

A família dos burgueses é naturalmente eliminada com a eliminação desse
seu complemento, e ambos desaparecem com o desaparecimento do capital.

Vocês censuram"-nos querer suprimir a exploração dos filhos pelos pais?
Nós confessamos esse crime.

Mas, dizem vocês, nós suprimimos as relações mais íntimas, à medida que
colocamos a educação social no lugar da doméstica.

E a educação de vocês também não está determinada pela sociedade? Não
está determinada pelas relações sociais, em cujo âmbito vocês educam,
pela ingerência mais ou menos direta ou indireta da sociedade, por meio
da escola etc.? Os comunistas não inventam o influxo da sociedade sobre
a educação; eles apenas modificam o seu caráter, eles subtraem a
educação à influência da classe dominante.

O palavrório burguês sobre família e educação, sobre a íntima relação de
pais e filhos torna"-se tanto mais repugnante quanto mais todos os
laços familiares, em consequência da grande indústria, são rompidos
para os proletários, e as suas crianças transformadas em simples
artigos de comércio e instrumentos de trabalho.

Mas vocês, comunistas, querem introduzir a comunidade das mulheres,
grita em coro, aos nossos ouvidos, a burguesia inteira.

O burguês enxerga em sua mulher um mero instrumento de produção. Ele
ouve dizer que os instrumentos de produção devem ser explorados
comunitariamente, e é natural que não consiga pensar outra coisa senão
que o destino do sistema de comunidade irá atingir igualmente as
mulheres.

Ele não imagina que se trata precisamente de suprimir a posição das
mulheres enquanto meros instrumentos de produção.

De resto, nada mais ridículo do que o espanto altamente moralista dos
nossos burgueses diante da comunidade oficial de mulheres pretensamente
proposta pelos comunistas. Os comunistas não precisam introduzir a
comunidade de mulheres, ela existiu quase sempre.

Os nossos burgueses, não satisfeitos em ter à sua disposição as mulheres
e as filhas dos seus proletários, para não falar da prostituição
oficial, encontram supremo divertimento em seduzir mutuamente suas
esposas.

O casamento burguês é, na realidade, a comunidade das esposas.
Poder"-se"-ia, no máximo, censurar os comunistas que, em lugar de
uma comunidade de mulheres hipocritamente ocultada, queiram
introduzir uma oficial, franca. De resto, entende"-se de imediato que,
com a supressão das atuais relações de produção, também a comunidade de
mulheres delas derivada, isto é, a prostituição oficial e não oficial
desaparece.

Além disso, foi censurado aos comunistas que eles queriam abolir a
pátria, a nacionalidade.

Os operários não têm pátria. Não se pode tirar deles o que não
possuem. Na medida em que o proletariado deve primeiramente conquistar
o domínio político, erigir"-se em classe nacional,\footnote{ Na edição de 
1888 consta aqui: “em classe dirigente da nação”. [N.~do~T.]} constituir"-
se ele mesmo enquanto nação, o próprio
proletariado é também nacional, ainda que de forma alguma no sentido da
burguesia.

As segregações nacionais e os antagonismos entre povos já vão desaparecendo
mais e mais com o desenvolvimento da burguesia, a liberdade de
comércio, o mercado mundial, a uniformidade da produção industrial
e as correspondentes relações de vida.

O domínio do proletariado os fará desaparecer ainda mais. Ação
unificada, pelo menos dos países civilizados, é uma das primeiras
condições de sua libertação.

À proporção que a exploração de um indivíduo pelo outro é suprimida,
suprime"-se a exploração de uma nação pela outra.

Com o antagonismo de classes no interior da nação, cai a postura hostil
das nações entre si.

As acusações contra o comunismo levantadas de pontos de vista
religiosos, filosóficos e ideológicos em geral, não merecem discussão
mais minuciosa.

Será necessária uma percepção profunda para entender que, com as
relações de vida dos homens, com os seus relacionamentos sociais, com a
sua existência social, também se modificam as suas representações, as
suas concepções e os seus conceitos  ---  modifica"-se, em uma palavra,
também a sua consciência?

Que outra coisa prova a história das ideias senão que a produção
intelectual se reconfigura com a produção material? As ideias
dominantes de uma época foram sempre tão somente as ideias da classe
dominante.

Fala"-se de ideias que revolucionaram toda uma sociedade; com isto,
apenas profere"-se o fato de que, no interior da velha sociedade,
formaram"-se os elementos de uma nova sociedade, que a dissolução das
velhas ideias caminha passo a passo com a dissolução das velhas
relações de vida.

Quando o mundo antigo estava em processo de desmoronamento, as religiões
antigas foram vencidas pela religião cristã. Quando as ideias cristãs
sucumbiam no século \textsc{xviii} às ideias iluministas, a sociedade feudal
travava a sua luta de morte com a burguesia então revolucionária. As
ideias de liberdade de consciência e de religião expressavam apenas a
dominação da livre concorrência no âmbito do saber.

“Mas”, dir"-se"-á, “ideias religiosas, morais, filosóficas, políticas,
jurídicas etc.~modificam"-se, entretanto, no decorrer do
desenvolvimento histórico. A religião, a moral, a filosofia, a
política, o direito sempre se preservaram no decorrer dessas mudanças.

Além disso, existem verdades eternas, como liberdade, justiça etc.,
comuns a todas as condições sociais. O comunismo, porém, abole as
verdades eternas, ele abole a religião, a moral, em vez de
configurá"-las de novo; ele contraria, portanto, todos os
desenvolvimentos históricos até o presente.”

A que se reduz essa acusação? A história de toda a sociedade até os dias
de hoje moveu"-se no interior de antagonismos de classes, que nas
diferentes épocas foram configurados de maneira diferente.

Mas não importa a forma que esses antagonismos tenham assumido, a
exploração de uma parte da sociedade pela outra é um fato comum a todos
os séculos passados. Não admira, por isso, que a consciência social de
todos os séculos, a despeito de toda a multiplicidade e variedade,
mova"-se no âmbito de certas formas comuns, em formas de consciência
que só se dissolvem plenamente com o desaparecimento completo do
antagonismo de classes.

A revolução comunista é a ruptura mais radical com as relações de
propriedade tradicionais; não admira que no curso de seu
desenvolvimento se rompa da maneira mais radical com as ideias
tradicionais.

Mas deixemos as investidas da burguesia contra o comunismo.

Já vimos acima que o primeiro passo na revolução operária é a elevação
do proletariado à condição de classe dominante, a conquista da
democracia.

O proletariado utilizará o seu domínio político para subtrair pouco a
pouco à burguesia todo o capital, para centralizar todos os
instrumentos de produção nas mãos do Estado, isto é, do proletariado
organizado como classe dominante, e para multiplicar o mais rapidamente
possível a massa das forças produtivas.

De início, isso naturalmente só pode acontecer por meio de intervenções
despóticas no direito de propriedade e nas relações de produção
burguesas, portanto, através de medidas que economicamente parecem
insuficientes e insustentáveis, mas que no curso do movimento
transcendem o seu próprio âmbito e serão inevitáveis como meios para o
revolucionamento do modo de produção em seu conjunto.

Naturalmente, essas medidas serão diferentes de acordo com os diferentes
países.

Para os países mais desenvolvidos, contudo, as seguintes medidas poderão
ser postas em prática de uma forma um tanto geral:

\begin{enumerate}
\item Expropriação da propriedade fundiária e emprego da renda fundiária
para despesas \mbox{estatais.} 

\item Pesado imposto progressivo. 

\item Abolição do direito de herança.

\item Confisco da propriedade de todos os emigrantes e insurrecionados. 

\item Centralização do crédito nas mãos do Estado por meio de um banco
nacional com capital estatal e monopólio exclusivo.

\item Centralização do sistema de transportes nas mãos do Estado. 

\item Multiplicação das fábricas nacionais, dos instrumentos de produção;
arroteamento e melhoria, segundo um plano comunitário, de grandes
extensões de terra. 

\item Obrigatoriedade de trabalho para todos; constituição de exércitos
industriais, especialmente para a agricultura. 

\item Unificação dos setores da agricultura e da indústria; atuação no
sentido da eliminação gradual da diferença entre cidade e campo. 

\item Educação pública e gratuita para todas as crianças. Eliminação do
trabalho infantil em fábricas na sua forma atual. Unificação da
educação com a produção material etc.
\end{enumerate}

Desaparecidas as diferenças de classes no curso do desenvolvimento e
concentrada toda a produção nas mãos dos indivíduos associados, então o
poder público perde o caráter político. O poder político em sentido
próprio é o poder organizado de uma classe para a opressão de uma
outra. Se, na luta contra a burguesia, o proletariado se unifica
necessariamente em classe, se ele converte"-se em classe dominante
mediante uma revolução e, como classe dominante, suprime à força as
velhas relações de produção, então ele estará suprimindo, com essas
relações de produção, as condições de existência do antagonismo de
classes, suprimindo as classes em geral e, desse modo, a sua própria
dominação enquanto classe.

No lugar da velha sociedade burguesa com as suas classes e os seus
antagonismos de classes surge uma associação na qual o livre
desenvolvimento de cada um é a condição para o livre desenvolvimento de
todos.

\section{Literatura socialista e comunista}

\vspace*{1em}
\subsection{O socialismo reacionário}

\subsubsection{O socialismo feudal}
Em consonância com a sua posição histórica, as aristocracias
francesa e inglesa estavam fadadas a escrever panfletos contra a
moderna sociedade burguesa. Na revolução francesa de julho de 1830, no
movimento reformista inglês, essas aristocracias mais uma vez				
sucumbiram ao odiado arrivista. Não se podia dizer mais que se tratava
de uma luta política séria. Restou"-lhes apenas a luta literária. Mas
também no âmbito da literatura, o velho palavrório da época da
Restauração\footnote{ Não é a Restauração Inglesa de 1660--1689 que se tem em mente, mas sim a
Restauração Francesa de 1814--1830. [Nota de F.~Engels para a edição
inglesa de 1888.]}
tornou"-se impossível. Para despertar simpatias, a
aristocracia precisou aparentar ter perdido de vista os seus interesses
e formular a sua acusação à burguesia somente no interesse da
classe operária explorada. Ela preparou assim a satisfação de
poder entoar invectivas ao seu novo senhor e sussurrar"-lhe aos
ouvidos profecias mais ou menos sinistras.

Dessa maneira surgiu o socialismo feudal, em parte canto de lamento, em
parte pasquim, em parte ressonância do passado, em parte ameaça do
futuro, por vezes atingindo com suas sentenças amargas,
espirituosamente dilacerantes, a burguesia em pleno coração, mas
atuando sempre de maneira cômica em sua total incapacidade de
compreender a marcha da história moderna.

Os aristocratas fizeram com que o saco de esmolas proletário
tremulasse em suas mãos como bandeira, para ajuntar o povo atrás de si.
Mas toda vez que seguia os aristocratas, o povo avistava em seu
traseiro os velhos brasões feudais e dispersava"-se com sonoras e
irreverentes
gargalhadas.\footnote{ Essa imagem dos “velhos brasões feudais”
estampados no traseiro dos aristocratas foi provavelmente inspirada
pelo \textit{epos} satírico de Heinrich Heine\textit{ Alemanha. Um
conto maravilhoso de inverno} (Cf.~nota 7 da p.~\pageref{7}). No terceiro capítulo, Heine
descreve os sentimentos que o acometem ao avistar, em Aachen, os
militares prussianos com os seus pontiagudos morriões de aço: “Isso
lembra a Idade Média, tão bela / Cavaleiros de nobre coração / Que
trazem no peito a fidelidade / E no traseiro portam o brasão”. [N.~do~T.]}

Uma parte dos legitimistas franceses e a Jovem Inglaterra levaram a
público esse espetáculo.

Quando os feudais comprovam que o seu modo de exploração estava
configurado de forma diferente da exploração burguesa, eles esquecem
apenas que exploraram sob circunstâncias e condições inteiramente
diversas. Quando demonstram que sob o seu domínio não existiu o
proletariado, se esquecem apenas de que essa mesma burguesia
moderna foi um rebento necessário de sua ordem social.

De resto, eles dissimulam tão pouco o caráter reacionário de sua crítica,
que a sua principal acusação contra a burguesia consiste justamente em
afirmar que, sob o regime burguês, desenvolve"-se uma classe que irá
mandar pelos ares toda a velha ordem social.

O que censuram à burguesia, mais do que gerar um proletariado em geral,
é o fato de que ela gera um proletariado revolucionário.

Por isso, na práxis política participam de todas as represálias
violentas contra a classe operária e na vida comum acomodam"-se, a
despeito de todo o seu palavrório enfatuado, em colher os pomos
dourados\footnote{ Na edição de 1888: “pomos que caíram da árvore da indústria”. [N.~do~T.]} e em trocar fidelidade, amor, honra, pela barganha com lã,
beterraba e
aguardente.\footnote{ Isto se refere principalmente à Alemanha, onde 
a nobreza rural e a classe dos \textit{Junker} cultivam por conta própria, 
por intermédio de seus administradores, uma grande parte de suas terras, 
e, ao lado disso, são ainda grandes produtores de açúcar de beterraba e 
de aguardente de batata. Os aristocratas ingleses, mais ricos, ainda não desceram 
a tanto; mas também sabem como compensar a queda dos rendimentos mediante 
a cessão de seus nomes a fundadores de sociedades acionárias de reputação 
mais ou menos duvidosa. [Nota de F.~Engels para a edição inglesa de 1888.]}

Do mesmo modo como o clérigo sempre andou de mãos dadas com o feudal,
assim também o socialismo clerical anda de mãos dadas com o socialismo
feudal.

Nada mais fácil do que dar ao ascetismo cristão um verniz socialista. O
cristianismo também não clamou contra a propriedade privada, o
casamento, o Estado? E em seu lugar não pregou a caridade e a
mendicância, o celibato e a mortificação da carne, a vida monástica e a
Igreja? O socialismo cristão é apenas a água benta com que o clérigo
abençoa a irritação do aristocrata.

\subsubsection{O socialismo pequeno"-burguês}

A aristocracia feudal não é a única classe derrubada pela
burguesia, cujas condições de vida definharam e pereceram na moderna
sociedade burguesa. O estamento medieval dos burgueses das
paliçadas\footnote{ \textit{Mittelalterliches Pfahlbürgertum},
no original (Cf.~nota 11 na p.~\pageref{8}). [N.~do~T.]} e o estamento dos pequenos camponeses
foram os precursores da moderna burguesia. Nos países industrial e
comercialmente menos desenvolvidos, essa classe ainda continua a
vegetar ao lado da burguesia ascendente.

Nos países em que a moderna civilização se desenvolveu,
formou"-se uma nova classe de pequeno"-burgueses, a qual oscila entre
o proletariado e a burguesia e está sempre se reformulando enquanto
parcela complementar da sociedade burguesa  ---  uma classe cujos membros
vão sendo arrastados constantemente para o proletariado e, com o
desenvolvimento da grande indústria, veem inclusive chegar o momento em
que desaparecerão por completo, enquanto parcela autônoma, da sociedade
moderna, e serão substituídos no comércio, na manufatura, na
agricultura, por supervisores de trabalho e por
criados.\footnote{ \textit{Domestiken}, no original. [N.~do~T.]}

Em países como a França, em que a classe camponesa perfaz bem mais do
que a metade da população, foi natural que escritores que se alinhavam
com o proletariado e contra a burguesia aplicassem à sua crítica do
regime burguês o padrão dos pequeno"-burgueses e pequenos camponeses,
tomando assim o partido dos operários a partir do ponto de vista da
pequena"-burguesia. Constituiu"-se, dessa maneira, o socialismo
pequeno"-burguês. Sismondi é o cabeça dessa literatura não apenas para
a França, mas também em relação à Inglaterra.

Esse socialismo dissecou com extrema perspicácia as contradições
existentes nas modernas relações de produção. Ele desvendou os
embelezamentos hipócritas dos economistas. Demonstrou, de maneira
irrefutável, os efeitos destrutivos da maquinaria e da divisão do
trabalho, a concentração dos capitais e da propriedade fundiária, a
superprodução, as crises, a necessária derrocada dos pequeno"-burgueses
e camponeses, a miséria do proletariado, a anarquia na produção, as
desproporções gritantes na distribuição da riqueza, a guerra industrial
de extermínio entre as nações, a dissolução dos velhos costumes, das
velhas relações familiares, das velhas nacionalidades.

Em seu teor positivo, contudo, esse socialismo quer, ou restabelecer os
velhos meios de produção e de circulação, e, com estes, as velhas
relações de propriedade e a velha sociedade, ou então forçar os
modernos meios de produção e de circulação a entrar novamente no quadro
das velhas relações de propriedade, as quais foram arrebentadas,
tiveram de ser arrebentadas por aqueles. Em ambos os casos, ele é
reacionário e utópico ao mesmo tempo.

Sistema corporativo na manufatura e economia patriarcal no campo, esta é
a sua última palavra.

Em seu desenvolvimento posterior, essa tendência perdeu"-se num
covarde coro de lamentações.\footnote{ Na edição de 1888: “Por fim, quando os obstinados fatos 		
históricos espantaram toda a embriaguez da autoilusão, essa forma de
socialismo degenerou em um lamentável coro de lamentações”. [N.~do~T.]}

\subsubsection{O socialismo alemão ou “verdadeiro”}

A literatura comunista e socialista da França, que nasceu sob a pressão
de uma burguesia dominante e é a expressão literária da luta contra
esse domínio, foi introduzida na Alemanha numa época em que a burguesia
estava começando a sua luta contra o absolutismo feudal.

Filósofos alemães, semifilósofos e beletristas apoderaram"-se
avidamente dessa literatura, e esqueceram apenas que, com a emigração
daqueles escritos franceses, não haviam migrado ao mesmo tempo para a
Alemanha as relações de vida francesas. Diante das relações alemãs, a
literatura francesa perdeu todo significado prático imediato e assumiu
uma aparência meramente literária. Foi forçoso aparecer como
especulação ociosa sobre a realização da essência humana. Para os
filósofos alemães do século \textsc{xvii}, as reivindicações da primeira
Revolução Francesa possuíam assim o sentido único de ser reivindicações
da “razão prática” em geral, e as manifestações de vontade por parte da
burguesia revolucionária francesa significavam aos seus olhos as leis
da vontade pura, da vontade, como esta tem de ser, da vontade
verdadeiramente humana.

O trabalho exclusivo dos literatos alemães consistiu em colocar as novas
ideias francesas em harmonia com a sua velha consciência filosófica ou,
antes, apropriar"-se das ideias francesas a partir de seu
posicionamento filosófico.

Essa apropriação aconteceu da mesma maneira pela qual geralmente se
apropria de uma língua estrangeira, pela tradução.

É sabido como os monges recobriam manuscritos em que estavam registradas
as obras clássicas da velha era pagã com insípidas histórias católicas
de santos. Os literatos alemães procederam de forma inversa com a
literatura francesa profana. Escreviam o seu disparate filosófico no
verso do original francês. Escreviam, por exemplo, no verso da crítica
francesa das relações monetárias, “alienação da essência humana”, atrás
da crítica francesa do Estado burguês escreviam “superação do domínio
do geral abstrato” etc.

A inserção sub"-reptícia desse palavrório filosófico nos desdobramentos
franceses, batizavam"-na “filosofia da ação”, “socialismo verdadeiro”,
“ciência alemã do socialismo”, “fundamentação filosófica do socialismo”
etc.

Desse modo, a literatura socialista"-comunista francesa foi formalmente
emasculada. E uma vez que, em mãos alemãs, ela deixou de expressar a
luta de uma classe contra a outra, o alemão ficou consciente de ter
superado a “unilateralidade francesa”, de ter representado, em vez de
necessidades verdadeiras, a necessidade da verdade, e, em vez dos
interesses do proletário, os interesses da essência humana, do homem de
uma maneira geral, do homem que não pertence a nenhuma classe, que de
modo algum pertence à realidade, que pertence apenas ao céu nebuloso da
fantasia filosófica.

Esse socialismo alemão, que recebeu os seus canhestros exercícios
escolares com tanta seriedade e solenidade e os alardeou de forma tão
charlatanesca, foi perdendo pouco a pouco sua inocência pedante.

A luta da burguesia alemã, notadamente da prussiana, contra os feudais e
a realeza absoluta  ---  em uma palavra, o movimento liberal  ---  tornou"-se
mais séria.

Ofereceu"-se assim ao “verdadeiro” socialismo a desejada oportunidade
de contrapor as reivindicações socialistas ao movimento político, de
lançar os anátemas tradicionais contra o liberalismo, o Estado
representativo, a concorrência burguesa, a liberdade de imprensa
burguesa, o direito burguês, a liberdade e a igualdade burguesas, e pregar
diante da massa popular que ela não tem nada a ganhar com esse
movimento burguês mas, antes, tudo a perder. O socialismo alemão se
esqueceu há tempo de que a crítica francesa, da qual ele era o eco sem
espírito, pressupunha a moderna sociedade burguesa, com as
correspondentes condições materiais de vida e a constituição política
adequada, pressupostos esses que na Alemanha ainda se tratava de
conquistar.

Ele servia aos governos absolutistas alemães, com o seu séquito de
clérigos, mestres"-escolas, nobres rurais e burocratas, como oportuno
espantalho contra a burguesia que estava em ameaçadora ascensão.

Ele constituía o complemento adocicado às amargas chibatadas e balas de
espingarda com que esses mesmos governos tratavam os levantes operários
alemães.

Se de tal maneira o socialismo “verdadeiro” tornou"-se uma arma
na mão dos governos contra a burguesia alemã, ele também representou,
de maneira imediata, um interesse reacionário, o interesse dos
atrasados burgueses alemães, da “burguesia das paliçadas”.\footnote{ \textit
{Pfahlbürgerschaft}, no original. Nessa passagem, a expressão é empregada
em sentido figurado (Cf.~nota 11 na p.~\pageref{8}). Na edição de 1888 encontra"-se substituída 
por “filisteus”. [N.~do~T.]} Na Alemanha, a pequena"-burguesia, 
proveniente do século \textsc{xvi} e desde esse tempo despontando aqui de forma sempre 
variada, constitui a efetiva base social das condições vigentes.

Sua manutenção é a manutenção das condições vigentes na Alemanha. Do
domínio industrial e político da burguesia, ela teme a derrocada certa,
por um lado, em consequência da concentração do capital, por outro lado,
pelo advento de um proletariado revolucionário. O socialismo
“verdadeiro” pareceu"-lhe matar dois coelhos de uma só cajadada. Ele
dissemina"-se como uma epidemia.

A roupagem, tecida de especulativas teias de aranha, bordada com flores
da retórica e da beletrística, impregnada de sufocante orvalho
sentimental, essa extravagante roupagem na qual os socialistas alemães
envolveram seu punhado de esquálidas “verdades eternas”, apenas
intensifica a aceitação da sua mercadoria entre esse público.

O socialismo alemão, por seu turno, foi reconhecendo cada vez mais sua
missão de ser o representante tonitruante dessa atrasada “burguesia das
paliçadas”.

Ele proclamava a nação alemã como sendo a nação normal e o
filisteu alemão como sendo o homem normal. A cada baixeza deste, ele
dava um sentido oculto, mais elevado, um sentido socialista, no qual
essa baixeza significava o seu contrário. Ele chegou às últimas
consequências ao postar"-se diretamente contra a tendência “rudimentar
e destrutiva” do comunismo e anunciar a sua superioridade apartidária
sobre todas as lutas de classes. Com muito poucas exceções, tudo o que,
de tais escritos pretensamente socialistas e comunistas, circula na
Alemanha pertence ao âmbito dessa literatura suja e enervante.\footnote{ A 
tempestade revolucionária de 1848 varreu do mapa toda essa sórdida 
tendência e estragou o prazer de seus defensores em continuar se envolvendo
com o socialismo. Principal representante e tipo clássico dessa tendência é
o senhor Karl Grün. [Nota de F.~Engels para a edição alemã de 1890.]}

\subsection{O socialismo conservador ou burguês}

Uma parcela da burguesia deseja corrigir as mazelas sociais para
assegurar a continuidade da sociedade burguesa.

Pertencem a essa parcela: economistas, filantropos, humanitários,
reformadores da situação das classes trabalhadoras, organizadores de
beneficências, protetores de animais, fundadores de ligas
antialcoólicas, tacanhos reformistas das mais variadas espécies. E
também esse socialismo burguês foi elaborado em sistemas completos.

Mencionemos, como exemplo, a \textit{Filosofia da miséria}, de
Proudhon.

Os burgueses socialistas querem as condições de vida da moderna
sociedade sem as lutas e os perigos que necessariamente decorrem delas.
Eles querem a sociedade vigente, mas subtraindo os elementos que a
revolucionam e a dissolvem. Eles querem a burguesia sem o proletariado.
A burguesia, naturalmente, representa para si mesma o mundo em que
domina como sendo o melhor dos mundos. O socialismo dos burgueses
elabora essa representação consoladora em um semissistema ou em um
sistema completo. Quando exorta o proletariado a concretizar os seus
sistemas e entrar na nova Jerusalém, então ele só exige no fundo que o
proletariado permaneça na sociedade atual, mas se desfaça das
representações hostis que faz desta.

Uma segunda forma desse socialismo, menos sistemática porém mais
prática, busca tirar a disposição da classe operária para qualquer
movimento revolucionário, demonstrando que não é essa ou aquela
transformação política que lhe poderá ser proveitosa, mas tão somente
uma transformação das relações materiais de vida, das relações
econômicas. Contudo, por transformação das relações materiais de vida,
esse socialismo não entende de maneira alguma a abolição das relações
burguesas de produção, a qual só é possível pela via revolucionária,
mas sim melhorias administrativas, que se processam no terreno dessas
relações de produção e, portanto, nada alteram na relação entre capital
e trabalho assalariado, mas, no melhor dos casos, diminuem para a
burguesia os custos do seu domínio e simplificam a sua gestão do
Estado.

Esse socialismo dos burgueses só alcança a sua expressão correspondente
quando se converte em mera figura retórica.

Livre comércio!  ---  no interesse da classe trabalhadora; proteções
alfandegárias!  ---  no interesse da classe trabalhadora; prisões em
sistema de celas!  ---  no interesse da classe trabalhadora: eis a ultima
palavra do socialismo dos burgueses, a única palavra levada a sério.

O socialismo da burguesia consiste justamente na afirmação de que os
burgueses são burgueses  ---  no interesse da classe trabalhadora.

\subsection[O socialismo e o comunismo crítico"-utópicos]{O socialismo e o comunismo crítico"-utópicos}

Não vamos falar aqui da literatura que em todas as grandes revoluções
modernas expressou as reivindicações do proletariado. (Escritos de
Babeuf etc.)

As primeiras tentativas do proletariado no sentido de fazer valer seu
próprio interesse de classe num tempo de agitação geral, no período da
derrubada da sociedade feudal, fracassaram necessariamente em face da
configuração pouco desenvolvida do próprio proletariado e da carência
das condições materiais para a sua libertação, as quais são justamente
o produto da época burguesa. A literatura revolucionária, que
acompanhou essas primeiras movimentações do proletariado, é
necessariamente reacionária em seu conteúdo. Ela ensina um ascetismo
geral e um igualitarismo grosseiro.

Os sistemas propriamente socialistas e comunistas, os sistemas
de Saint"-Simon, Fourier, Owen etc., surgem no primeiro período, pouco
desenvolvido, da luta entre proletariado e burguesia, que expusemos
acima (ver Burguesia e
Proletariado\footnote{ Trata"-se do segmento \textsc{i} “Burgueses e
proletários”. [N.~do~T.]}).

É verdade que os inventores desses sistemas enxergam tanto o antagonismo
das classes como a eficácia dos elementos dissolventes na própria
sociedade dominante. Mas não divisam, no campo do proletariado, nenhuma
autonomia histórica, nenhum movimento político que lhe seja peculiar.

Uma vez que o desenvolvimento do antagonismo das classes caminha passo a
passo com o desenvolvimento da indústria, eles tampouco encontram as
condições materiais para a libertação do proletariado, e procuram assim
por uma ciência social, por leis sociais, no intuito de criar essas
condições.

No lugar da atividade social é preciso entrar a sua própria atividade
inventiva, no lugar das condições históricas de libertação entram
condições fantásticas, no lugar da organização do proletariado em
classe, que vai se processando gradualmente, entra uma organização da
sociedade engendrada por eles mesmos. A história universal que está por
vir dissolve"-se, para eles, na propaganda e na execução prática de seus
planos sociais.

É verdade que estão conscientes de representarem em seus planos o
interesse da classe trabalhadora como sendo a classe mais sofredora. O
proletariado existe para eles somente sob esse ponto de vista da classe
mais sofredora.

Mas a forma pouco desenvolvida da luta de classes, assim como a sua
própria situação de vida, tem por consequência o fato de se julgarem
muito acima daquele antagonismo das classes. Querem melhorar a situação
de vida de todos os membros da sociedade, mesmo a dos mais bem
situados. Por isso apelam continuamente ao conjunto da sociedade, sem
distinção  ---  de preferência, inclusive, à classe dominante. Basta
compreender o seu sistema para reconhecê"-lo como o melhor plano
possível da melhor sociedade possível.

Rejeitam, por isso, toda ação política, notadamente toda ação
revolucionária, querem alcançar a sua meta por via pacífica e tentam
abrir caminho para o novo evangelho social através de pequenos
experimentos, que naturalmente malogram, através da força do exemplo.

Numa época em que o proletariado ainda se encontra muito pouco
desenvolvido, época em que, portanto, ele mesmo concebe de modo ainda
fantástico a sua própria situação, essa descrição fantástica da
sociedade futura brota de seu primeiro anseio intuitivo por uma
reconfiguração geral da sociedade.

Todavia, os escritos socialistas e comunistas comportam também elementos
críticos. Atacam todos os fundamentos da sociedade vigente. Forneceram,
por isso, um material extremamente valioso para o esclarecimento dos
operários. Suas sentenças positivas sobre a sociedade futura, por
exemplo, supressão do antagonismo entre cidade e campo, supressão da
família, do lucro privado, do trabalho assalariado, o anúncio da
harmonia social, a conversão do Estado em uma mera administração da
produção  ---  todas essas suas sentenças exprimem meramente a eliminação
do antagonismo das classes, antagonismo que está começando agora a se
desenvolver e que aqueles escritos conhecem tão somente em sua primeira
indeterminação amorfa. Por isso, essas mesmas sentenças têm um sentido
ainda puramente utópico.

O significado do socialismo e do comunismo crítico"-utópicos
está na razão inversa de seu desenvolvimento histórico. Na mesma medida
em que a luta de classes se desenvolve e se configura, essa elevação
fantástica sobre tal luta, esse combate fantástico movido contra esta,
perde todo valor prático, toda justificativa teórica. Se, portanto, os
artífices desses sistemas também foram revolucionários em muitos
aspectos, os seus discípulos constituem, a cada vez, seitas reacionárias.
Aferram"-se, em face do contínuo desenvolvimento histórico do
proletariado, às velhas concepções dos mestres. Procuram, por isso, de
maneira consequente, embotar novamente a luta de classes e conciliar as
oposições. Continuam a sonhar com a realização, em regime experimental,
de suas utopias sociais, instituição de falanstérios isolados, fundação
de \textit{home}"-colônias, implantação de uma pequena
Icária\footnote{ \textit{Home}"-colônias (colônias no
interior) chama Owen às suas modelares sociedades comunistas.
Falanstério era o nome dos palácios sociais planejados por Fourier.
Icária chamava"-se o utópico país da fantasia cujas instituições
comunistas Cabet descreveu. [Nota de F.~Engels para a edição alemã de
1890.]} --- edição em formato reduzido\footnote{ A expressão “edição em formato reduzido” corresponde, no original, a \textit{Duodezausgabe}, termo derivado do
latim \textit{duodecimus}: significa algo de dimensões ridiculamente
diminutas. [N.~do~T.]} da nova Jerusalém  ---  e para a construção de todos esses 
castelos de Espanha precisam apelar à filantropia dos corações e dos
endinheirados bolsos burgueses. Paulatinamente vão caindo na categoria
dos socialistas reacionários e conservadores acima retratados, e
distinguem"-se destes tão somente por um pedantismo mais sistemático,
pela crença supersticiosa e fanática nos efeitos miraculosos de sua
ciência social.

Por isso eles se opõem com exasperação a todo movimento político
dos operários que só pôde originar"-se a partir de uma descrença cega
no novo evangelho.\footnote{ Marx e Engels fazem aqui um trocadilho com a expressão crença ou fé “cega”: os seguidores tardios, retardatários de Owen, 
Fourier, Cabet, Saint"-Simon etc. contrapõem"-se encarniçadamente aos movimentos 
operários que puderam constituir"-se justamente porque ignoraram 
os seus obsoletos preceitos utópicos  ---  ou votaram"-lhes uma descrença 
“cega”, absoluta. [N.~do~T.]}

Os adeptos de Owen na Inglaterra e de Fourier na França reagem, lá,
contra os cartistas, aqui, contra os reformistas.

\section{Posição dos comunistas em relação aos diversos partidos oposicionistas}

Pelo exposto no segmento \textsc{ii}, fica evidente a relação dos comunistas 
com os partidos operários já constituídos, a sua relação, portanto, 
com os cartistas na Inglaterra e os reformadores agrários na América do
Norte.

Eles lutam para alcançar os objetivos e os interesses imediatos
da classe operária, mas, no movimento presente, representam, ao mesmo
tempo, o futuro do movimento. Na França, os comunistas aliam"-se ao
partido social"-democrata\footnote{ O partido que era então representado 
no parlamento por Ledru"-Rollin, na literatura por Louis Blanc e na 
imprensa diária pelo \textit{Réforme.} O nome “social"-democrata” 
significava, entre esses seus inventores, uma seção do partido 
democrático ou republicano com coloração mais ou menos socialista. 
[Nota de F.~Engels para a edição inglesa de 1888.] [\ldots] Era, portanto, abissalmente diferente da atual social"-democracia 
alemã. [Nota de F.~Engels para a edição alemã de 1890.]}
contra a burguesia conservadora e radical, sem que por isso abram mão do direito 
de se relacionar criticamente com a
fraseologia e as ilusões legadas pela tradição revolucionária.

Na Suíça, apoiam os radicais, sem deixar de reconhecer que esse partido
comporta elementos contraditórios, em parte socialistas democráticos no
sentido francês, em parte burgueses radicais.

Entre os poloneses, os comunistas apoiam o partido que faz de uma
revolução agrária condição de libertação nacional, o mesmo partido que
gerou a insurreição cracoviana de 1846.

Na Alemanha, logo que a burguesia entra em cena revolucionariamente, o 
partido comunista luta ombro a ombro com a burguesia contra a monarquia absoluta, 
a propriedade rural feudal e a “pequena"-burguesice”.\footnote{ “Pequena"-burguesice” traduz aqui o substantivo (não dicionarizado) \textit{Kleinbürgerei}, 
cujo sentido pejorativo advém do sufixo “ei” (correspondente ao português “ice”). [N.~do~T.]}

Mas em momento algum deixa de elaborar nos operários a consciência mais clara possível a respeito da oposição hostil entre burguesia e
proletariado, para que os operários alemães possam converter de
imediato as condições sociais e políticas, produzidas necessariamente
pelo domínio burguês, em outras tantas armas voltadas contra a
burguesia, para que, depois da derrubada das classes reacionárias na
Alemanha, comece imediatamente a luta contra a própria burguesia.

É em primeiro lugar para a Alemanha que os comunistas dirigem sua
atenção máxima, porque a Alemanha está às vésperas de uma revolução
burguesa e porque realiza esse revolucionamento sob as condições mais
avançadas da civilização europeia em geral e com um proletariado muito
mais desenvolvido do que tinha a Inglaterra no século \textsc{xvii} e a França
no século \textsc{xviii}, de maneira que a revolução burguesa alemã só pode
ser o prelúdio imediato de uma revolução proletária.

Os comunistas, numa palavra, apoiam por toda parte todo movimento
revolucionário contra as condições sociais e políticas vigentes.

Em todos esses movimentos, eles enfatizam a questão da propriedade, não
importa a forma mais ou menos desenvolvida que esta possa ter assumido,
como sendo a questão fundamental do movimento.

Os comunistas, por fim, trabalham em toda parte pela união e pelo
entendimento dos partidos democráticos de todos os países.

Os comunistas recusam"-se a dissimular suas visões e suas intenções.
Declaram abertamente que os seus objetivos só podem ser alcançados pela
derrubada violenta de toda a ordem social vigente até o presente. Que
tremam as classes dominantes em face de uma revolução comunista. Nela,
os proletários nada têm a perder senão as suas cadeias. Eles têm um
mundo a ganhar.

Proletários de todos os países, uni"-vos!



