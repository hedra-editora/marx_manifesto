\pagestyle{plain}

\chapter*{Introdução\smallskip\subtitulo{Diagnóstico do mundo\break moderno}}
\addcontentsline{toc}{chapter}{Introdução, \textit{por Ricardo Musse}}

\begin{flushright}
\textsc{ricardo musse}
\end{flushright}

\noindent{}O \textit{Manifesto do Partido Comunista}, redigido na antevéspera da
revolução de 1848, foi um dos primeiros textos a apresentar o mundo
moderno, que se descortinava no horizonte posterior à Revolução
Francesa de 1789, como uma sociedade perpassada por conflitos
insuperáveis. O diagnóstico de Marx\footnote{Ainda que a autoria do \textit{Manifesto} 
seja historicamente atribuída a Marx e Engels, este último foi responsável somente pela 
elaboração de seus primeiros dois rascunhos, o \textit{Esboço de confissão de fé comunista} e \textit{Princípios do comunismo}. 
A redação do \textit{Manifesto}, portanto, é obra somente de Marx, ainda que ele tenha incorporado 
elementos dos rascunhos de Engels. [\textsc{n.\,e.}]} destoa radicalmente dos
prognósticos de sua época --- tanto no campo da filosofia da história,
marcado pelas ideias de Condorcet ou mesmo de Hegel, como no da
economia política, na linhagem que se estende de Adam Smith a Ricardo,
ou ainda no da emergente sociologia positivista, configurada na França
por Saint"-Simon e Auguste Comte --- que traçavam, de modo quase
apologético, um cenário para o futuro da humanidade consubstanciado na
perspectiva de ampliação da liberdade, na expectativa de superação dos
conflitos políticos e sociais e no ideal de um mundo em paz perpétua.

Marx apresenta o \textit{Manifesto do Partido Comunista} como uma
autoexposição do comunismo. Trata"-se, em suas palavras, de
enunciar a versão do comunismo segundo os comunistas, procurando opor
``ao conto da carochinha sobre o espectro do comunismo um manifesto do próprio partido''.
Conjugado a essa tentativa de exposição teórica das premissas de um
movimento político que, mal entrara em cena já invocava para si o papel
de protagonista, Marx compôs um diagnóstico da modernidade que
esquematiza, em linhas gerais, tópicos que só serão desenvolvidos
detalhadamente em obras posteriores, particularmente no conjunto de
textos projetados pelo próprio Marx como uma \textit{crítica da economia
política} e cuja formulação mais acabada consiste em \textit{O
Capital}.

Essa súmula do mundo moderno, um pequeno esboço de história universal,
que o \textit{Manifesto do Partido Comunista} apresenta em poucas
páginas, dotadas de um impressionante poder de compreensão e síntese,
constitui a primeira aplicação e exposição pública da concepção
materialista que Marx e Engels haviam desenvolvido em um manuscrito
redigido em 1845--46, \textit{A ideologia alemã.} Após uma tentativa
fracassada de publicação, esse texto, segundo a terminologia deles
próprios, foi \textit{abandonado à crítica roedora dos ratos}. O
\textit{Manifesto}, além de retomar, sob a forma de drásticos resumos,
passagens inteiras desse manuscrito, concretiza a ideia, ali apenas
enunciada, de uma história que não prescinde das diversas perspectivas:
econômicas, sociais e políticas.

A \textit{teoria da história}, aí desenvolvida, propõe"-se a combater o ponto
de vista de um \textit{assim chamado desenvolvimento geral do espírito humano}
por meio da ênfase na observação das relações materiais. Seu fio
condutor foi posteriormente condensado por Marx nos seguintes termos: 

\begin{quote} 
Na produção social da própria vida, os homens contraem relações
determinadas, necessárias e independentes de sua vontade, relações de
produção estas que correspondem a uma etapa determinada de
desenvolvimento das suas forças produtivas materiais. A totalidade
destas relações de produção forma a estrutura econômica da sociedade, a
base real sobre a qual se levanta uma superestrutura jurídica e
política, e à qual correspondem formas sociais determinadas de
consciência. O modo de produção da vida material condiciona o processo
em geral de vida, social, político e espiritual. Não é a consciência
dos homens que determina o seu ser, mas, ao contrário, é seu ser social
que determina sua consciência. Em certa etapa de seu desenvolvimento,
as forças produtivas materiais da sociedade entram em contradição com
as relações de produção existentes  ou, o que nada mais é do que a sua
expressão jurídica, com as relações de propriedade dentro das quais
aquelas até então se tinham movido.\footnote{Karl Marx. Prefácio a
\textit{Para a crítica da economia política}.}
\end{quote} 

A trava no desenvolvimento das forças produtivas, enfatizando sua
contradição com as relações sociais existentes, manifesta"-se sob a
forma de crises. As relações próprias do mundo burguês, a partir de
determinado momento, teriam se tornado estreitas demais para
conter a riqueza ``colossal'', ``adormecidas no seio do trabalho social'',
que a burguesia despertou por meio da exploração do mercado mundial.
Medidas protelatórias, segundo Marx, apenas preparariam crises ``cada
vez mais amplas e poderosas''.

Tendo em vista esse cenário, o \textit{Manifesto} faz uma dupla aposta.
Primeiro, sustenta a hipótese, que se revelou verdadeira, de que a
crise levaria a uma revolução social que varreria do mapa europeu os
velhos regimes. Equivocou"-se, porém, na previsão de que o processo
produtivo capitalista já se desdobrara o suficiente para tornar
possível uma vitória definitiva do proletariado. Em 1850, Marx reconhece, no último artigo de\textit{ As lutas de classes na França 
(1848--1850)}, que a perspectiva de uma continuação do processo
revolucionário fora inviabilizada pela retomada, após a crise de 1847,
da prosperidade industrial.

O desfecho das revoluções de 1848, cristalizado na França pelo golpe de
Estado de Luís Bonaparte, em dezembro de 1851, que levou Marx a se
exilar na Inglaterra, onde se dedicou, por longos anos, à redação de
uma ``crítica da economia política'' --- corporificada em \textit{O
Capital} --- alterou sua visão acerca do papel político da burguesia.
Sua capacidade em se acomodar, quando preciso, com setores da
aristocracia fundiária e com a burocracia monárquica, desfez a
impressão, amplificada pela descrição do \textit{Manifesto}, de que se
tratava de uma classe movida por um impulso revolucionário, capaz de
``criar o mundo à sua imagem''. Depois desse desfecho, Marx passa a
enfatizar o papel contrarrevolucionário da classe burguesa, atenta (e
temerosa) à possibilidade de revolução social que traz para o primeiro
plano, como previram, seu conflito com o proletariado.

\section*{Diagnóstico do mundo moderno}

Encontramos no \textit{Manifesto} a combinação, quase sempre
inextricável, de uma exposição concisa, que se propõe a apresentar as
coordenadas econômicas, sociais e políticas do mundo moderno, com a
apresentação de uma teoria do comunismo que não se exime, entre outros
pontos, de estabelecer uma plataforma política do proletariado para uma
revolução que Marx e Engels julgavam iminente e que de fato se
desencadeou pouco menos de um mês após sua redação.

Marx efetiva assim uma das exigências essenciais do debate filosófico e
intelectual da década de 1840, abordando, de forma pertinente, a
questão do presente histórico. Mas não se trata apenas de uma revolução
na filosofia. Além das contribuições no campo da sociologia, a teoria
do conflito e das classes sociais e da economia --- embora aqui ainda
esteja ausente um ponto central de seu arcabouço futuro, a teoria
marxista do valor ---, o \textit{Manifesto} inaugura ainda a
interpretação econômica da história e a moderna teoria da política.

O gesto inaugural ou a introdução de avanços em disciplinas
aparentemente tão díspares --- que dificilmente poderá, por conta da
superespecialização hoje vigente no trabalho intelectual, ser repetido
por um outro livro --- explica"-se facilmente por um círculo virtuoso.
Marx renovou a história porque conhecia bem economia, revolucionou a
política porque conhecia a história como poucos, reinterpretou
criticamente a economia graças aos seus conhecimentos de política e de
história etc.

Não se pode dizer o mesmo, porém, do processo de disseminação que tornou
o marxismo um fenômeno mundial a partir da última década do século \textsc{xix}.
Como a divulgação das ideias de Marx se fez, prioritariamente, pela via da
esquematização, a difusão acarretou o empobrecimento tanto do conteúdo
quanto do método. Este empobrecimento não foi acarretado apenas pela
redistribuição do legado de Marx em partes e disciplinas estanques por
obra do anseio enciclopédico da época e pela posterior incorporação, em
separado, de algumas descobertas do marxismo pelo mundo acadêmico
burguês. O próprio Engels, apenas cinco anos depois da morte de Marx,
acrescentou ao \textit{Manifesto}, na edição inglesa de 1888 e, depois,
na edição alemã, uma série de notas explicativas, presentes em todas as
edições e traduções posteriores, que dissociam conceito e história. A
primeira nota, por exemplo, adendo ao título do primeiro segmento,
\textit{Burgueses e proletários}, define logicamente estas duas classes por
sua posição em relação à propriedade dos meios de produção. Já o
\textit{Manifesto} expõe esses conceitos por meio de uma síntese da
história moderna que destaca o processo de formação de cada classe e a
conexão entre elas, o antagonismo que as envolve numa luta
ininterrupta.

Dito em termos drásticos, do então manuscrito, posteriormente editado
como \textit{A ideologia alemã}, bem como de seus inúmeros estudos
sobre história, Marx tomou como pressuposto no \textit{Manifesto}
apenas um esquema mínimo, a tese de que ``a história de todas as
sociedades até o presente é a história das lutas de classes''.
Trata"-se, portanto, de trazer para o centro do relato da história
humana o conflito, a ``luta ininterrupta, ora dissimulada, ora aberta''
entre oprimidos e opressores.

O \textit{Manifesto}, apesar do tom panfletário inerente aos seus
objetivos práticos, políticos e pedagógicos, mantém a postura crítica
em relação à filosofia da história --- alçada então à condição de parte
nobre da especialização filosófica ---, explicitamente abordada em
\textit{A ideologia alemã.} Em lugar de estabelecer uma teleologia para
o desenvolvimento geral da espécie humana, Marx, analisando em bloco o
destino do mundo moderno, apenas aponta duas tendências, ou
possibilidades, extraídas da observação do passado histórico,
procurando evitar recair na ideia de uma necessidade inerente ao
espírito ou em alguma forma de determinismo: ``uma reconfiguração
revolucionária de toda a sociedade'' ou uma ``derrocada comum das classes
em luta''.

Na descrição de Marx, a ``moderna sociedade burguesa [\ldots{}] não aboliu os
antagonismos de classe'', mas, antes, colocou novas classes, novas
condições de opressão, novas formas e estruturas de luta. Tal situação
foi sintetizada na tese de que, no mundo moderno, haveria uma
simplificação dos antagonismos de classe. Em suas palavras, 

\begin{quote} 
a nossa época, a época da burguesia, caracteriza"-se, contudo, pelo
fato de ter simplificado os antagonismos de classes. A sociedade toda
cinde"-se, mais e mais, em dois grandes campos inimigos, em duas
grandes classes diretamente confrontadas: burguesia e proletariado.\footnote{Ver p.\,\pageref{1}.}
\end{quote} 

Esta tese, muitas vezes compreendida literalmente, prestou"-se a uma
série de equívocos. No âmbito do marxismo da Segunda Internacional (1889--1916),
transformou"-se em dogma, que não deixou de ser rebatido pelos
clássicos da sociologia alemã. A simplificação, apresentada por Marx
no contexto de uma indicação sobre o destino global do mundo moderno,
como possibilidades abertas pelo predomínio, enquanto \textit{sujeitos
históricos}, da burguesia ou do proletariado e, portanto, como uma
descrição sucinta da modernidade, transformou"-se, nas mãos de Karl
Kautsky e Eduard Bernstein, um afirmando outro negando, --- em flagrante
contradição com o espírito e a letra do texto de Marx --- na famosa e
polêmica \textit{tendência à polarização}. Sustentavam que a diferenciação e a
diversidade de classes iria futuramente ceder lugar a um cenário social
em que os indivíduos se classificariam literalmente como burgueses
ou como proletários.

\section*{O mundo da burguesia}

Marx apresenta a burguesia como sujeito histórico, por meio de uma breve
exposição da história moderna. Não distingue o conceito de sua
exposição, ao contrário de Engels, e da maior parte da tradição
marxista, que, como vimos, acrescentou, nas edições posteriores à morte
de Marx, notas em que procura forjar para a burguesia, e também para o
proletariado, uma definição formal, separando em campos distintos o
lógico e o histórico e concedendo primazia ao primeiro.

Em sua apresentação da burguesia, Marx associa o desenvolvimento
histórico"-social dessa classe, e principalmente sua constituição como
força política --- logo, como sujeito histórico ---, a uma série de
acontecimentos que marcaram a gênese e os desdobramentos do mundo
moderno. Poderíamos destacar aí, quatro momentos principais: 

\begin{enumerate}
\item A
descoberta da América e a circum"-navegação da África, levando em conta o
que esse episódio significou em termos de multiplicação dos meios de
troca ou de impulso fornecido simultaneamente ao comércio, à indústria
e à navegação
\item O surgimento da manufatura, isto é, a concentração
de produtores, privados da posse de seus instrumentos, em grande número
sob o mesmo teto. Com isso, a divisão do trabalho entre as diversas
corporações cede lugar à divisão do trabalho dentro de cada oficina\footnote{Tema que será exaustivamente desenvolvido no livro primeiro de
\textit{O Capital}.}
\item A implantação da Revolução Industrial, quando
a introdução da maquinaria, e posteriormente do vapor como energia
motriz, altera significativamente a produção industrial. Com isso, a
manufatura, nos centros mais desenvolvidos, é substituída pela \textit{grande
indústria moderna}. Com ela, com a usina, com o mundo da fábrica,
surgiram os ``milionários industriais, os chefes de exércitos
industriais inteiros, os burgueses modernos''
\item A partir das
premissas desenvolvidas pela implantação em larga escala da \textit{grande
indústria} cria"-se o \textit{mercado mundial}, impulsionando de forma
inaudita a indústria, o comércio, os transportes
\end{enumerate}

Nesse painel, a burguesia moderna é apresentada como o produto de um
longo processo, de uma série de revoluções nos meios de produção,
transportes e comunicação, por meio do qual ela se desenvolve,
economicamente, multiplicando seus capitais e, politicamente,
empurrando para o segundo plano as demais classes opressoras. Assim,
Marx adverte que, ainda que as demais classes opressoras não sejam
suprimidas, doravante quem dá as cartas nos rumos do desenvolvimento
histórico e na luta política é a burguesia.

A trajetória política da burguesia segue, quase passo a passo, em
sintonia e correspondência, as modificações sociais e históricas do
mundo moderno. Seu itinerário na fase do mercado mundial compreende a passagem de classe oprimida
à condição de classe opressora, de estrato social oprimido sob o
domínio dos senhores feudais ao domínio político exclusivo.

Diga"-se de passagem que a famosa frase de Marx que atribui ao poder
estatal moderno a mera condição de ``comissão que administra os negócios
comuns do conjunto da classe burguesa'', em geral contestada como
reducionismo --- ou como um diagnóstico equivocado da vida política,
inerente a uma carência de pensamento político que seria própria do
marxismo ---, quando devidamente situada em seu contexto social e
histórico, não parece tão fora de propósito assim.

Hoje, após o fim do interregno marcado pela presença do Estado de
bem"-estar social, assistimos, como na época do \textit{Manifesto
} --- sobretudo porque o último quartel do século \textsc{xx} seria marcado, como a
primeira metade do século \textsc{xix}, pelo predomínio do \textit{mercado mundial} ---,
uma inesperada redução do poder de pressão das demais classes sobre as
modalidades e a direção adotadas na condução do Estado.

\section*{Dialética da modernidade}

No \textit{Manifesto do Partido Comunista}, Marx apresenta,
pela primeira vez, o mundo burguês como uma unidade contraditória entre
fatores dinâmicos e invariância estática. O paradoxo de uma sociedade
que não pode existir sem revolucionar continuamente os instrumentos de
produção e, com eles, o conjunto das relações sociais, é próprio do
mundo moderno. Enquanto os antigos modos de produção assentavam"-se, à
maneira de uma tradição, na manutenção e na conservação de relações fixas
e cristalizadas, a sociedade burguesa se reproduz, mantendo"-se
idêntica somente ao preço de uma contínua transformação que,
acarretando a obsolescência e uma incessante destruição de toda
estrutura de produção existente em um determinado momento, subverte
inclusive o cenário histórico e político.

Por razões conjunturais, no \textit{Manifesto}, Marx privilegiou, nesse entrelaçamento, o aspecto dinâmico, a constância
da transitoriedade, materializado na frase"-emblema: ``Tudo que é
sólido desmancha no ar''.\footnote{Ver nota\,\ref{2} na p.\,\pageref{2}.} Muito do interesse e parte da recepção desse
texto explicam"-se por essa ênfase. Em períodos de estabilização e
consolidação do capital, seja entre 1850 e 1870 ou no quase meio século
que se estende de 1950 a 1989, o marxismo voltou"-se para a
compreensão da estática imanente à dinâmica social, concebendo a
sociedade como uma segunda natureza e debruçando"-se sobre o
sempre"-igual de fenômenos como o fetichismo da mercadoria. Hoje, no
entanto, quando o engessamento do capitalismo --- provocado então por uma
conjunção especial de fatores: conflito entre blocos e Guerra Fria,
estabelecimento nos países centrais de um Estado de bem"-estar social,
predomínio incontestável da hegemonia norte"-americana --- parece ter
chegado ao fim, muito do que se diz no \textit{Manifesto} volta a ter
uma inesperada atualidade.

Essa dinâmica, própria da modernidade, é apresentada e desdobrada no
\textit{Manifesto} sob a forma de uma dupla expansão que ocorre
simultaneamente, embora em direções e sobre domínios diferenciados. Uma
vez que Marx não nomeia explicitamente esse processo, tomaremos a
liberdade de denominar esses movimentos de \textit{expansão intensiva e extensiva}.

A descrição do papel \textit{eminentemente revolucionário} desempenhado pela
burguesia na história moderna pode ser concebida como uma história dos
movimentos do agente histórico dessa expansão, o que explica, entre
outras coisas, a forte carga irônica dessas passagens, muitas vezes
interpretadas como uma espécie de apologia da burguesia.

Primeiro, Marx descreve a \textit{expansão intensiva}, isto é, os movimentos
segundo os quais o capitalismo extravasa o campo das relações puramente
econômicas, espraiando"-se para outras esferas da vida social. Esse
processo caracteriza"-se por uma inaudita mercantilização e reificação
de todo o domínio social, atingindo inclusive o âmago da subjetividade.
Diz ele: 

\begin{quote} 
Onde quer que a burguesia tenha chegado ao poder, ela destruiu todas as
relações feudais, patriarcais, idílicas. Ela rompeu impiedosamente os
variados laços feudais que atavam o homem ao seu superior natural,
não deixando nenhum outro laço entre os seres humanos senão o interesse
nu e cru, senão o insensível ‘pagamento à vista’. Ela afogou os
arrepios sagrados do arroubo religioso, do entusiasmo cavalheiresco, da
plangência do filisteísmo burguês, nas águas gélidas do cálculo
egoísta. Ela dissolveu a dignidade pessoal em valor de troca, e no
lugar das inúmeras liberdades atestadas em documento ou valorosamente
conquistadas, colocou \textit{uma} única inescrupulosa liberdade de
comércio. A burguesia, em uma palavra, colocou no lugar da exploração
ocultada por ilusões religiosas e políticas a exploração aberta,
desavergonhada, direta, seca. A burguesia despojou de sua auréola
sagrada todas as atividades até então veneráveis, contempladas com
piedoso recato. Ela transformou o médico, o jurista, o clérigo, o
poeta, o homem das ciências, em trabalhadores assalariados, pagos por ela.
A burguesia arrancou às relações familiares o seu comovente véu
sentimental e as reduziu a pura relação monetária.\footnote{Ver p.\,\pageref{3}.}
\end{quote} 

Mas, ao mesmo tempo em que salienta o predomínio de relações mercantis e
da reificação sobre o conjunto da vida social, Marx detecta outro
movimento expansionista, próprio do mundo moderno, que denominamos
\textit{expansão extensiva}, caracterizado pela colonização, ou melhor, pela
penetração capitalista sobre áreas e regiões econômicas ainda não
capitalistas --- o que abrange desde áreas do mundo rural, situadas
próximas ao centro do capitalismo, até os territórios
pré"-capitalistas, situados nos confins do planeta. Diz ele:

\begin{quote} 
Por meio da exploração do mercado mundial, a burguesia configurou, de
maneira cosmopolita, a produção e o consumo de todos os países. Para
grande pesar dos reacionários, ela subtraiu à indústria o solo nacional
em que esta tinha os pés. As antiquíssimas indústrias nacionais foram
aniquiladas e ainda continuam sendo aniquiladas diariamente. São
sufocadas por novas indústrias, cuja introdução se torna uma questão
vital para todas as nações civilizadas, são sufocadas por indústrias que não mais
processam matérias"-primas nativas, mas sim matérias"-primas próprias
das zonas mais afastadas, e cujos produtos são consumidos não apenas no
próprio país, mas simultaneamente em todas as partes do mundo. No lugar
das velhas necessidades, satisfeitas pelos produtos nacionais, surgem
novas necessidades, que requerem, para a sua satisfação, os produtos dos
mais distantes países e climas. No lugar da velha autossuficiência e
do velho isolamento locais e nacionais, surge um intercâmbio em
todas as direções, uma interdependência múltipla das nações. [...] Os
módicos preços de suas mercadorias são a artilharia pesada com que ela
põe abaixo todas as muralhas da China, com que ela constrange à
capitulação mesmo a mais obstinada xenofobia dos bárbaros. Ela obriga
todas as nações que não queiram desmoronar a apropriar"-se do modo de
produção da burguesia; ela as obriga a introduzir em seu próprio meio a
assim chamada civilização, isto é, a tornarem"-se burguesas. Em uma
palavra, ela cria para si um mundo à sua própria imagem e semelhança.\footnote{Ver p.\,\pageref{5}.}
\end{quote} 

O agente dessa dupla expansão, a burguesia --- apresentado ao mesmo tempo
em que a descrição desse processo histórico ---, passa a ser concebido,
pelo menos ao longo do \textit{Manifesto}, como um autêntico sujeito
histórico, isto é, como uma classe dotada da capacidade de
autodeterminação. Esse traço da classe burguesa pode ser concebido, por
um lado, como uma espécie de explicação materialista para a filosofia
do Idealismo alemão, particularmente para o conceito hegeliano de
Espírito, no qual, como se sabe, concebe"-se uma substância que é ao
mesmo tempo sujeito. Mas, por outro lado, tal descrição antecipa o
modo como o próprio Marx concebe o proletariado: como um \textit{sujeito
histórico}, isto é, como um agrupamento coletivo, autônomo, capaz de se
autodeterminar e influir sobre os destinos do mundo moderno.\looseness=-1

A \textit{expansão extensiva}, hoje denominada globalização, vincula"-se de
forma mais estreita, no \textit{Manifesto}, com a fase do mercado
mundial, que corresponde, no campo da história política, à conjuntura
posterior à Revolução Francesa. Por \textit{mercado mundial}, Marx designa
tanto uma forma de concentração industrial --- na primeira metade do
século \textsc{xix}, o surgimento da \textit{grande indústria}, equivalente, no último
quartel do século \textsc{xx}, às fusões de conglomerados e o subsequente
predomínio de gigantescas empresas transnacionais --- quanto o domínio
exclusivo do poder pela burguesia --- na época do \textit{Manifesto}, o
recém"-implantado Estado constitucional representativo, equivalente,
hoje, ao predomínio de políticas estatais neoliberais e a difusão de
preceitos visando à redução ao mínimo do Estado.

Mas, ao mesmo tempo, tanto a \textit{expansão extensiva} como a \textit{expansão
intensiva} são assinaladas, já no texto do \textit{Manifesto}, como
expedientes a que a burguesia recorre para tentar superar as crises do
capitalismo. Vejamos o que diz Marx:

\begin{quote} 
Por quais meios a burguesia supera as crises? Por um lado, pelo
extermínio forçado de grande parte das forças produtivas; por outro
lado, pela conquista de novos mercados e pela exploração mais metódica
dos antigos mercados.\footnote{Ver p.\,\pageref{9}.}
\end{quote} 

Se parece evidente que o termo \textit{conquista de novos mercados} corresponde
ao momento dinâmico que denominamos \textit{expansão extensiva}, o outro termo
utilizado, \textit{exploração metódica dos antigos mercados}, deve ser
entendido como uma reformulação dos meios e das formas de produção --- que
abrange desde a tecnologia empregada na produção até as formas de
manejo da mão de obra no interior do processo produtivo --- que não
pode ser levada adiante sem a derrubada de obstáculos --- jurídicos,
culturais etc. --- e por uma intensificação da padronização específica da
economia capitalista sobre as demais esferas do mundo social.

\section*{A marca da contradição}

Para apresentar o núcleo da contradição imanente ao mundo moderno,
matriz do conflito social ininterrupto que perpassa o capitalismo, Marx
recorre, novamente, apenas a um esquema interpretativo mínimo que toma
por fio condutor da exposição. Da mesma forma que, na sociedade feudal,
em certo estágio de desdobramento dos meios de produção e de circulação
sobre cujas bases a burguesia se formou, as relações feudais de
propriedades foram percebidas como um entrave, no mundo moderno, o
potencial inscrito nas forças produtivas e no trabalho social não pode
se desenvolver plenamente nas atuais condições de produção e segundo as
relações sociais estabelecidas.

As relações burguesas de produção e de propriedade, em suma, a moderna
sociedade burguesa assemelha"-se ao ``feiticeiro que já não consegue mais
dominar os poderes subterrâneos que invocou''. Uma contraprova disso
seria a própria história da indústria e do comércio --- na perspectiva de
Marx ---, marcada há decênios pela ``revolta das modernas forças
produtivas contra as modernas relações de produção, contra as relações
de propriedade que constituem as condições vitais da burguesia e de sua
dominação''.

\textls[-5]{Outros sintomas dessa contradição, além das cada vez mais frequentes
sublevações do proletariado, seriam as recorrentes crises comerciais,
no \textit{Manifesto} denominadas \textit{epidemias da superprodução}. O
melhor argumento a favor da tese de que as forças produtivas que estão
à disposição da sociedade estão emperradas pelas relações de
propriedade burguesas, seria fornecido, no \textit{Manifesto}, pelos
expedientes utilizados pela burguesia para superar tais crises, em
especial o extermínio forçado de forças produtivas. Cabe acrescentar
que, mesmo a utilização dos outros expedientes, a conquista a qualquer
preço de novos mercados, ou a intensificação da exploração metódica dos
antigos mercados costumam, como se sabe, desembocar em guerras ou em
rebeliões.}

A exposição do proletariado, no decorrer do \textit{Manifesto}, embora
possa ser remetida ao quadro histórico"-econômico próprio do mundo
moderno, não privilegia as mediações econômicas, mas, antes, a história
de sua formação política. De modo geral, Marx salienta que, na mesma
medida em que a burguesia, ou melhor, o capital também se desdobra, o
proletariado se desenvolve. De tal forma que,

\begin{quote} 
as armas com as quais a burguesia derruiu o feudalismo voltam"-se agora
contra a própria burguesia. Mas a burguesia não forjou apenas as armas que
lhe trazem a morte; ela produziu também os homens que portarão essas
armas --- os operários modernos, os \textit{proletários}.\footnote{Ver p.\,\pageref{4}.}
\end{quote} 

Embora, ao longo da tradição marxista, em especial nas linhagens da
Segunda (1889--1916) e da Terceira Internacional (1919--43), a noção de classe operária tenda
a ser compreendida dentro de um figurino estrito, associado em geral,
ao trabalho na \textit{grande indústria}, no \textit{Manifesto do partido
comunista}, Marx determina, de modo genérico, o proletariado como
aqueles que ``só subsistem enquanto encontram trabalho, e só encontram
trabalho enquanto seu trabalho aumenta o capital''. 

Esse modo de compreender a inserção social do proletariado destaca a
submissão do mundo do trabalho à lógica econômica do mercado: os
``operários, que têm de se vender um a um, são uma mercadoria como
qualquer outro artigo de comércio e, por isso, igualmente expostos a
todas as vicissitudes da concorrência, a todas as oscilações do
mercado''.

Em nossos termos, podemos então dizer que a dinâmica da \textit{expansão
intensiva}, própria do capitalismo, afeta o proletariado no âmago da sua
inserção --- e integração --- social, enquanto força de trabalho,
consolidando"-se como um dos principais obstáculos à sua organização e
formação política. Esse processo foi destacado por Marx como uma forma
de reificação típica da situação do trabalho no capitalismo. Primeiro,
o trabalhador perde sua autonomia, pela via da expansão da maquinaria e
pela ampliação da divisão de trabalho no interior do processo de
produção, tornando"-se quase um mero acessório da máquina; mas
sobretudo, ele passa doravante a estar submetido, no interior da
fábrica, ao \textit{despotismo}: ``Eles não apenas são servos da classe
burguesa, do Estado burguês; diariamente e a cada hora eles são
escravizados pela máquina, pelo supervisor e, sobretudo, por cada um
dos fabricantes burgueses''.

A \textit{expansão extensiva}, por outro lado, reponde sistematicamente àquele
processo que Marx irá designar, em \textit{O Capital}, como \textit{acumulação
primitiva}, cria um incessante contingente de \textit{empregáveis} que
intensifica a concorrência entre os trabalhadores. Ao reduzir os custos
do trabalho para o capital, esse excesso de oferta de mão de obra
acaba por empurrar os proventos da classe operária ao patamar mínimo,
configurado pelos valores necessários à mera reprodução da espécie.

A formação política do proletariado, descrita no \textit{Manifesto do
Partido Comunista}, afigura"-se, portanto, como uma forma de superação,
seja da alienação própria ao mundo do trabalho, seja dos resultados da
incessante concorrência entre os trabalhadores gerada pela recorrente
reprodução da mão de obra. Nesse sentido, um dos pressupostos
práticos do \textit{Manifesto}, a tarefa de contribuir para a
organização do proletariado em um partido político, não pode ser vista
como um objetivo descolado da necessidade de superar esses obstáculos.

Marx descreve, passo a passo, a constituição do proletariado como
\textit{sujeito histórico} por meio de uma série de percalços que moldaram sua
formação como agente político: 

\begin{quote} 
No início lutam os operários isolados, depois os operários de uma
fábrica, depois os operários de um ramo industrial, numa mesma região,
contra um burguês particular, que os explora diretamente. Eles dirigem os seus
ataques não apenas contra as relações de produção burguesas; eles os
dirigem contra os próprios instrumentos de produção; eles aniquilam as
mercadorias estrangeiras concorrentes, destroçam as máquinas, ateiam
fogo nas fábricas, buscam reconquistar a soterrada posição do
trabalhador medieval.\footnote{Ver p.\,\pageref{6}.}
\end{quote} 

Essa massa, ainda dispersa e fragmentada pela concorrência, só
se insere na luta política em favor da burguesia. Desse modo,
``combatem não os seus inimigos, mas sim os inimigos de seus inimigos''.
Mas, mesmo que toda vitória conquistada, nesse momento, seja afinal das
contas uma vitória da burguesia, é por meio de tais lutas que o
proletariado inicia seu aprendizado, encaminhando sua formação
propriamente política.

Só mais tarde, o poder social do proletariado deixa de ser algo latente
e se torna efetivo: ``com o desenvolvimento da indústria, não apenas
se multiplica o proletariado, este é agregado em massas cada vez maiores,
sua força cresce e torna"-se mais perceptível para ele mesmo''. A força
social do proletariado consuma"-se com sua organização em escala
nacional nos moldes de um agrupamento plenamente político. Essa forma
de associação, alicerçada em formas de organização que começam no
interior das fábricas, é impulsionada pela conjugação de uma série de
fatores, próprios da grande indústria --- e que hoje, no mundo da
desregulamentação e da flexibilização do trabalho, tendem cada vez mais
a desaparecer. Primeiro, destaca"-se a tendência à homogeneização do
proletariado, resultado da implantação da \textit{grande indústria},
acarretando uma diminuição da desigualdade no interior do proletariado
--- desigualdade essa que muito dificultou sua organização em momentos
anteriores. Essa tendência à generalização do conflito culmina em
formas de organização que constituem propriamente o proletariado como
classe, agente político, ou melhor, como \textit{sujeito histórico}.

Pode"-se considerar que a aposta de Marx, recorrente ao longo do
\textit{Manifesto do Partido Comunista}, atribui ao proletariado a
possibilidade de desempenhar na história papel equivalente ao exercido
pela burguesia. Vale dizer que, para Marx, a classe operária tem a
possibilidade de se posicionar como agente determinante do destino
histórico do mundo moderno. 

Esse engajamento da classe operária em um projeto de transformação
social não é apresentado como um resultado automático e necessário
decorrente das condições econômicas e sociais da sociedade burguesa.
Marx adverte o tempo todo que os incessantes esforços para organizar o
proletariado em partido político são, a cada instante, contrariados
pela concorrência entre os próprios operários, bem como pela reificação,
condições inerentes a sua situação no capitalismo.

Como vimos, com a \textit{grande indústria} ampliam"-se alguns fatores que
contribuem para intensificar a força social do proletariado. Uma vez
completa a sua formação política, a classe operária amplia sua
capacidade de liderança no conjunto configurado pelas diversas classes
que se defrontam com a burguesia --- os estratos médios, o pequeno
industrial, o pequeno comerciante, o artesão, o camponês etc. ---, chegando
inclusive a obter a adesão e o apoio esporádico de setores da própria
classe dominante, ameaçados em suas condições de vida. 

Todos esses fatores são, entretanto, contrabalançados pelos obstáculos
cristalizados na reificação e pela concorrência entre os trabalhadores.
A generalização da forma"-mercadoria, aquela dinâmica que chamamos de
\textit{expansão intensiva}, dificulta não só a afirmação do proletariado como
sujeito histórico, mas a própria reflexão acerca dos problemas
inscritos no cerne da sociedade capitalista, uma vez que a reificação,
originariamente atuante no mundo do trabalho, estende"-se para todas
as esferas da vida, isto é, para todos os setores da sociedade.

\chapter{Nota do tradutor}

\begin{flushright}
\textsc{marcus mazzari }
\end{flushright}

\noindent{}Esta tradução do \textit{Manifesto do Partido Comunista} foi feita a
partir do texto estabelecido no primeiro volume dos escritos selecionados de
Marx e Engels: \textit{Marx--Engels: Ausgewählte Schriften in 2
Bänden}, Dietz Verlag, Berlim, 1982.\footnote{Páginas 17--57.} Utilizou"-se
ainda o texto editado por Siegfried Landshut no volume \textit{Karl
Marx: Die Frühschriften --- Von 1837 bis zum Manifest der
kommunistischen Partei 1848}, Alfred Kröner Verlag, Stuttgart, 1971.\footnote{Páginas 525--560.} O tradutor gostaria de expressar o seu
agradecimento a Zenir Campos Reis pela leitura, acompanhada de valiosas
observações, da primeira versão deste texto. Registra ainda, igualmente
agradecido, que teve a oportunidade de discutir alguns passos desta
tradução com Alfredo Bosi.



