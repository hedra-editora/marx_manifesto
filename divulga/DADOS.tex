\titulo{Manifesto comunista}
\autor{Marx e Engels}  % Apenas sobrenome, se for o caso. Verificar capa.% 
\organizador{Tradução}{Marcus Mazzari}   %Conferir se é apenas {Organização}; {Organização e tradução} ou apenas {Tradução}%
\isbn{978-85-7715-191-2}			
\preco{14}   % Ex.: 14. Não usar ",00"%
\pag{118}   % Número de páginas
\release{\textsc{Manifesto comunista} (Manifest der Kommunistischen Partei, 1848),
publicado originalmente como “Manifesto do Partido Comunista”, foi encomendado
pela Liga dos Comunistas e publicado em 21 de fevereiro de 1848. É um dos textos		
mais influentes do mundo, expondo o programa da Liga, e contando com uma análise
da luta de classes, tanto a partir de uma perspectiva histórica, quanto
contemporânea, e que trata do período em que se estabelecia o capitalismo e,
consequentemente, a burguesia como classe dominante e sua luta permanente com o
proletariado, na Europa do século XIX. O \textit{Manifesto}, ainda que tenha
incorporado elementos de outros pensadores, constituiu as bases da teoria sobre
as classes sociais no capitalismo e a luta de classes, fundamentando os
princípios do marxismo. Ainda que a autoria do \textit{Manifesto} seja
historicamente atribuída a Marx e Engels, este último foi responsável somente
pela elaboração de seus primeiros rascunhos e a redação foi realizada por Marx.

\noindent\textsc{Karl Marx} (Trier, 1818--Londres, 1883), intelectual revolucionário
alemão e um dos principais ideólogos do comunismo moderno, atuou como
filósofo, economista, historiador e teórico político. Estudou na Universidade de
Bonn e depois na Universidade de Berlim, onde Hegel era professor. Em Berlim,
participou do Clube dos Doutores e, perdendo o interesse pelo direito, passou ao
estudo da filosofia. Fez parte da esquerda hegeliana, em 1841 doutorou-se em
filosofia e, no ano seguinte, tornou-se redator-chefe da \textit{Gazeta Renana}
de Colônia. Ainda em 1842, conhece Engels e passa, posteriormente, a dirigir os
\textit{Anais Franco-Alemães} em Paris. Em 1843, vincula-se à Liga dos Justos,
futura Liga dos Comunistas, quando passa a estudar e escrever sobre economia política,
socialismo e história da França, aderindo ao socialismo. Expulso do território francês em
1845, se estabelece em Bruxelas onde escreve, junto com Engels, o
\textit{Manifesto Comunista}, quando é novamente expulso e parte para Colônia, fundando,
também com Engels, a \textit{Nova Gazeta Renana}. 
Estabelece-se finalmente em Londres, onde participa da Associação Internacional dos
Trabalhadores, ou “Primeira Internacional”.   % Aqui vai o nome do organizador e/ou tradutor. 

\noindent\textsc{Friedrich Engels} (Barmen, 1820--Londres, 1895), revolucionário alemão,
desenvolveu, junto com Marx, o chamado “socialismo científico”. Filho mais velho
de um industrial da tecelagem, viveu em Berlim e depois em Manchester, onde
conheceu Marx, em 1842, ao se envolver com o jornalismo radical e a política.
Ao voltar à Alemanha em 1844, passou por Paris, onde reencontrou Marx,
aproximando-se dele definitivamente. Em Bruxelas, auxiliou na formação da Liga
dos Comunistas. Morou em Colônia e participou como fundador da \textit{Nova
Gazeta Renana}. Em 1849, tomou parte de um levante no sul da Alemanha e, com seu
fracasso, voltou à Inglaterra. Em Manchester, volta a trabalhar na empresa de
seu pai e passa a sustentar Marx. 
Com a morte de Marx, trabalhou na preparação e na publicação dos dois últimos
volumes de \textit{O Capital}. Investiu seu tempo em outras produções teóricas e
teve significativa influência na social-democracia alemã.   % Aqui vai o nome do organizador e/ou tradutor. 

\noindent\textsc{Marcus Vinicius Mazzari} é professor de Teoria Literária na Universidade
de São Paulo. Traduziu para o português textos de Walter Benjamin, Bertolt
Brecht, Adelbert von Chamisso, Thomas Mann, Günter Grass, Goethe, entre outros.
Publicou ainda, entre outros trabalhos, \textit{Romance de formação em perspectiva histórica}
(Ateliê, 1999) e \textit{Die Danziger Trilogie von Günter Grass. Erzählen gegen
die Dämonisierung deutscher Geschichte} (Berlim, 1994).  % Se houver um prefaciador (que não seja o próprio tradutor ou organizador)

\noindent\textsc{Ricardo Musse} é professor no departamento de sociologia da Universidade
de São Paulo e editor do \textit{Jornal de Resenhas}.  % Se houver um prefaciador (que não seja o próprio tradutor ou organizador)

}


